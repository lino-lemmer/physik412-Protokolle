% Copyright © 2013 Martin Ueding <dev@martin-ueding.de>

% Copyright © 2012-2013 Martin Ueding <dev@martin-ueding.de>

% This is my general purpose LaTeX header file for writing German documents.
% Ideally, you include this using a simple ``% Copyright © 2012-2013 Martin Ueding <dev@martin-ueding.de>

% This is my general purpose LaTeX header file for writing German documents.
% Ideally, you include this using a simple ``% Copyright © 2012-2013 Martin Ueding <dev@martin-ueding.de>

% This is my general purpose LaTeX header file for writing German documents.
% Ideally, you include this using a simple ``\input{header.tex}`` in your main
% document and start with ``\title`` and ``\begin{document}`` afterwards.

% If you need to add additional packages, I recommend not doing this in this
% file, but in your main document. That way, you can just drop in a new
% ``header.tex`` and get all the new commands without having to merge manually.

% Since this file encorporates a CC-BY-SA fragment, this whole files is
% licensed under the CC-BY-SA license.

\documentclass[11pt, ngerman, fleqn, DIV=15, BCOR=2cm, headinclude]{scrartcl}

\usepackage{graphicx}

% Environment to quote the problem. Currently, this is just a new name for the
% quote environment.
\newenvironment{problem}{\begin{quote}\textsf{\textbf{Aufgabenstellung:}}\quad}{\end{quote}}

\setkomafont{caption}{\sf}
\setkomafont{captionlabel}{\usekomafont{caption}}

%%%%%%%%%%%%%%%%%%%%%%%%%%%%%%%%%%%%%%%%%%%%%%%%%%%%%%%%%%%%%%%%%%%%%%%%%%%%%%%
%                                Locale, date                                 %
%%%%%%%%%%%%%%%%%%%%%%%%%%%%%%%%%%%%%%%%%%%%%%%%%%%%%%%%%%%%%%%%%%%%%%%%%%%%%%%

\usepackage{babel}
\usepackage[iso]{isodate}

%%%%%%%%%%%%%%%%%%%%%%%%%%%%%%%%%%%%%%%%%%%%%%%%%%%%%%%%%%%%%%%%%%%%%%%%%%%%%%%
%                          Margins and other spacing                          %
%%%%%%%%%%%%%%%%%%%%%%%%%%%%%%%%%%%%%%%%%%%%%%%%%%%%%%%%%%%%%%%%%%%%%%%%%%%%%%%

\usepackage[parfill]{parskip}
\usepackage{setspace}
\usepackage[activate]{microtype}

\setlength{\columnsep}{2cm}

%%%%%%%%%%%%%%%%%%%%%%%%%%%%%%%%%%%%%%%%%%%%%%%%%%%%%%%%%%%%%%%%%%%%%%%%%%%%%%%
%                                    Color                                    %
%%%%%%%%%%%%%%%%%%%%%%%%%%%%%%%%%%%%%%%%%%%%%%%%%%%%%%%%%%%%%%%%%%%%%%%%%%%%%%%

\usepackage[usenames, dvipsnames]{xcolor}

\colorlet{darkred}{red!70!black}
\colorlet{darkblue}{blue!70!black}
\colorlet{darkgreen}{green!40!black}

%%%%%%%%%%%%%%%%%%%%%%%%%%%%%%%%%%%%%%%%%%%%%%%%%%%%%%%%%%%%%%%%%%%%%%%%%%%%%%%
%                         Font and font like settings                         %
%%%%%%%%%%%%%%%%%%%%%%%%%%%%%%%%%%%%%%%%%%%%%%%%%%%%%%%%%%%%%%%%%%%%%%%%%%%%%%%

% This replaces all fonts with Bitstream Charter, Bitstream Vera Sans and
% Bitstream Vera Mono. Math will be rendered in Charter.
\usepackage[charter, greekuppercase=italicized]{mathdesign}
\usepackage{beramono}
\usepackage{berasans}

% Bold, sans-serif tensors. This fragment is taken from “egreg” from
% http://tex.stackexchange.com/a/82747/8945 and licensed under `CC-BY-SA
% <https://creativecommons.org/licenses/by-sa/3.0/>`_.
\usepackage{bm}
\DeclareMathAlphabet{\mathsfit}{\encodingdefault}{\sfdefault}{m}{sl}
\SetMathAlphabet{\mathsfit}{bold}{\encodingdefault}{\sfdefault}{bx}{sl}
\newcommand{\tens}[1]{\bm{\mathsfit{#1}}}

% Bold vectors.
\renewcommand{\vec}[1]{\boldsymbol{#1}}

%%%%%%%%%%%%%%%%%%%%%%%%%%%%%%%%%%%%%%%%%%%%%%%%%%%%%%%%%%%%%%%%%%%%%%%%%%%%%%%
%                               Input encoding                                %
%%%%%%%%%%%%%%%%%%%%%%%%%%%%%%%%%%%%%%%%%%%%%%%%%%%%%%%%%%%%%%%%%%%%%%%%%%%%%%%

\usepackage[T1]{fontenc}
\usepackage[utf8]{inputenc}

%%%%%%%%%%%%%%%%%%%%%%%%%%%%%%%%%%%%%%%%%%%%%%%%%%%%%%%%%%%%%%%%%%%%%%%%%%%%%%%
%                         Hyperrefs and PDF metadata                          %
%%%%%%%%%%%%%%%%%%%%%%%%%%%%%%%%%%%%%%%%%%%%%%%%%%%%%%%%%%%%%%%%%%%%%%%%%%%%%%%

\usepackage{hyperref}
\usepackage{lastpage}

% This sets the author in the properties of the PDF as well. If you want to
% change it, just override it with another ``\hypersetup`` call.
\hypersetup{
	breaklinks=false,
	citecolor=darkgreen,
	colorlinks=true,
	linkcolor=darkblue,
	menucolor=black,
	pdfauthor={Martin Ueding},
	urlcolor=darkblue,
}

%%%%%%%%%%%%%%%%%%%%%%%%%%%%%%%%%%%%%%%%%%%%%%%%%%%%%%%%%%%%%%%%%%%%%%%%%%%%%%%
%                               Math Operators                                %
%%%%%%%%%%%%%%%%%%%%%%%%%%%%%%%%%%%%%%%%%%%%%%%%%%%%%%%%%%%%%%%%%%%%%%%%%%%%%%%

% AMS environments like ``align`` and theorems like ``proof``.
\usepackage{amsmath}
\usepackage{amsthm}

% Common math constructs like partial derivatives.
\usepackage{commath}

% Physical units.
\usepackage[output-decimal-marker={,}]{siunitx}

% Since I use mathdesign with italic uppercase greek characters, the Ohm unit will be displayed with an italic Ω by default. Units have to be roman, so this forces it the right way.
\DeclareSIUnit{\ohm}{$\Omegaup$}
\DeclareSIUnit{\division}{DIV}
\DeclareSIUnit{\voltss}{$\mathrm{V_{SS}}$}

% Word like operators.
\DeclareMathOperator{\acosh}{arcosh}
\DeclareMathOperator{\arcosh}{arcosh}
\DeclareMathOperator{\arcsinh}{arsinh}
\DeclareMathOperator{\arsinh}{arsinh}
\DeclareMathOperator{\asinh}{arsinh}
\DeclareMathOperator{\card}{card}
\DeclareMathOperator{\csch}{cshs}
\DeclareMathOperator{\diam}{diam}
\DeclareMathOperator{\sech}{sech}
\renewcommand{\Im}{\mathop{{}\mathrm{Im}}\nolimits}
\renewcommand{\Re}{\mathop{{}\mathrm{Re}}\nolimits}

% Fourier transform.
\DeclareMathOperator{\fourier}{\ensuremath{\mathcal{F}}}

% Roman versions of “e” and “i” to serve as Euler's number and the imaginary
% constant.
\newcommand{\ee}{\eup}
\newcommand{\eup}{\mathrm e}
\newcommand{\ii}{\iup}
\newcommand{\iup}{\mathrm i}

% Symbols for the various mathematical fields (natural numbers, integers,
% rational numbers, real numbers, complex numbers).
\newcommand{\C}{\ensuremath{\mathbb C}}
\newcommand{\N}{\ensuremath{\mathbb N}}
\newcommand{\Q}{\ensuremath{\mathbb Q}}
\newcommand{\R}{\ensuremath{\mathbb R}}
\newcommand{\Z}{\ensuremath{\mathbb Z}}

% Shape like operators.
\DeclareMathOperator{\dalambert}{\Box}
\DeclareMathOperator{\laplace}{\bigtriangleup}
\newcommand{\curl}{\vnabla \times}
\newcommand{\divergence}[1]{\inner{\vnabla}{#1}}
\newcommand{\vnabla}{\vec \nabla}

\newcommand{\half}{\frac 12}

% Unit vector (German „Einheitsvektor“).
\newcommand{\ev}{\hat{\vec e}}

% Scientific notation for large numbers.
\newcommand{\e}[1]{\cdot 10^{#1}}

% Mathematician's notation for the inner (scalar, dot) product.
\newcommand{\bracket}[1]{\left\langle #1 \right\rangle}
\newcommand{\inner}[2]{\bracket{#1, #2}}

% Placeholders.
\newcommand{\emesswert}{\del{\messwert \pm \messwert}}
\newcommand{\fehlt}{\textcolor{darkred}{Hier fehlen noch Inhalte.}}
\newcommand{\messwert}{\textcolor{blue}{\square}}
\newcommand{\punkte}{\phantom{xxxxx}}
\newcommand{\punktevon}[1]{\begin{flushright}/ #1\end{flushright}}

% Separator for equations on a single line.
\newcommand{\eqnsep}{,\quad}

% Quantum Mechanics
\usepackage{braket}

%%%%%%%%%%%%%%%%%%%%%%%%%%%%%%%%%%%%%%%%%%%%%%%%%%%%%%%%%%%%%%%%%%%%%%%%%%%%%%%
%                                  Headings                                   %
%%%%%%%%%%%%%%%%%%%%%%%%%%%%%%%%%%%%%%%%%%%%%%%%%%%%%%%%%%%%%%%%%%%%%%%%%%%%%%%

% This will set fancy headings to the top of the page. The page number will be
% accompanied by the total number of pages. That way, you will know if any page
% is missing.
%
% If you do not want this for your document, you can just use
% ``\pagestyle{plain}``.

\usepackage{scrpage2}

\pagestyle{scrheadings}
\automark{section}
\cfoot{\footnotesize{Seite \thepage\ / \pageref{LastPage}}}
\chead{}
\ihead{}
\ohead{\rightmark}
\setheadsepline{.4pt}

%%%%%%%%%%%%%%%%%%%%%%%%%%%%%%%%%%%%%%%%%%%%%%%%%%%%%%%%%%%%%%%%%%%%%%%%%%%%%%%
%                            Bibliography (BibTeX)                            %
%%%%%%%%%%%%%%%%%%%%%%%%%%%%%%%%%%%%%%%%%%%%%%%%%%%%%%%%%%%%%%%%%%%%%%%%%%%%%%%

\newcommand{\bibliographyfile}{../../central-bibtex/Central}
\bibliographystyle{apalike2}

%%%%%%%%%%%%%%%%%%%%%%%%%%%%%%%%%%%%%%%%%%%%%%%%%%%%%%%%%%%%%%%%%%%%%%%%%%%%%%%
%                                Abbreviations                                %
%%%%%%%%%%%%%%%%%%%%%%%%%%%%%%%%%%%%%%%%%%%%%%%%%%%%%%%%%%%%%%%%%%%%%%%%%%%%%%%

\newcommand{\dhabk}{\mbox{d.\,h.}}

%%%%%%%%%%%%%%%%%%%%%%%%%%%%%%%%%%%%%%%%%%%%%%%%%%%%%%%%%%%%%%%%%%%%%%%%%%%%%%%
%                                  Licences                                   %
%%%%%%%%%%%%%%%%%%%%%%%%%%%%%%%%%%%%%%%%%%%%%%%%%%%%%%%%%%%%%%%%%%%%%%%%%%%%%%%

\usepackage{ccicons}

\newcommand{\ccbysadetext}{%
	\begin{small}
		Dieses Werk bzw. Inhalt steht unter einer
		\href{http://creativecommons.org/licenses/by-sa/3.0/deed.de}{%
			Creative Commons Namensnennung - Weitergabe unter gleichen
		Bedingungen 3.0 Unported Lizenz}.
	\end{small}
}

\newcommand{\ccbysadetitle}{%
	Lizenz: \href{http://creativecommons.org/licenses/by-sa/3.0/deed.de}
	{CC-BY-SA 3.0 \ccbysa}
}
`` in your main
% document and start with ``\title`` and ``\begin{document}`` afterwards.

% If you need to add additional packages, I recommend not doing this in this
% file, but in your main document. That way, you can just drop in a new
% ``header.tex`` and get all the new commands without having to merge manually.

% Since this file encorporates a CC-BY-SA fragment, this whole files is
% licensed under the CC-BY-SA license.

\documentclass[11pt, ngerman, fleqn, DIV=15, BCOR=2cm, headinclude]{scrartcl}

\usepackage{graphicx}

% Environment to quote the problem. Currently, this is just a new name for the
% quote environment.
\newenvironment{problem}{\begin{quote}\textsf{\textbf{Aufgabenstellung:}}\quad}{\end{quote}}

\setkomafont{caption}{\sf}
\setkomafont{captionlabel}{\usekomafont{caption}}

%%%%%%%%%%%%%%%%%%%%%%%%%%%%%%%%%%%%%%%%%%%%%%%%%%%%%%%%%%%%%%%%%%%%%%%%%%%%%%%
%                                Locale, date                                 %
%%%%%%%%%%%%%%%%%%%%%%%%%%%%%%%%%%%%%%%%%%%%%%%%%%%%%%%%%%%%%%%%%%%%%%%%%%%%%%%

\usepackage{babel}
\usepackage[iso]{isodate}

%%%%%%%%%%%%%%%%%%%%%%%%%%%%%%%%%%%%%%%%%%%%%%%%%%%%%%%%%%%%%%%%%%%%%%%%%%%%%%%
%                          Margins and other spacing                          %
%%%%%%%%%%%%%%%%%%%%%%%%%%%%%%%%%%%%%%%%%%%%%%%%%%%%%%%%%%%%%%%%%%%%%%%%%%%%%%%

\usepackage[parfill]{parskip}
\usepackage{setspace}
\usepackage[activate]{microtype}

\setlength{\columnsep}{2cm}

%%%%%%%%%%%%%%%%%%%%%%%%%%%%%%%%%%%%%%%%%%%%%%%%%%%%%%%%%%%%%%%%%%%%%%%%%%%%%%%
%                                    Color                                    %
%%%%%%%%%%%%%%%%%%%%%%%%%%%%%%%%%%%%%%%%%%%%%%%%%%%%%%%%%%%%%%%%%%%%%%%%%%%%%%%

\usepackage[usenames, dvipsnames]{xcolor}

\colorlet{darkred}{red!70!black}
\colorlet{darkblue}{blue!70!black}
\colorlet{darkgreen}{green!40!black}

%%%%%%%%%%%%%%%%%%%%%%%%%%%%%%%%%%%%%%%%%%%%%%%%%%%%%%%%%%%%%%%%%%%%%%%%%%%%%%%
%                         Font and font like settings                         %
%%%%%%%%%%%%%%%%%%%%%%%%%%%%%%%%%%%%%%%%%%%%%%%%%%%%%%%%%%%%%%%%%%%%%%%%%%%%%%%

% This replaces all fonts with Bitstream Charter, Bitstream Vera Sans and
% Bitstream Vera Mono. Math will be rendered in Charter.
\usepackage[charter, greekuppercase=italicized]{mathdesign}
\usepackage{beramono}
\usepackage{berasans}

% Bold, sans-serif tensors. This fragment is taken from “egreg” from
% http://tex.stackexchange.com/a/82747/8945 and licensed under `CC-BY-SA
% <https://creativecommons.org/licenses/by-sa/3.0/>`_.
\usepackage{bm}
\DeclareMathAlphabet{\mathsfit}{\encodingdefault}{\sfdefault}{m}{sl}
\SetMathAlphabet{\mathsfit}{bold}{\encodingdefault}{\sfdefault}{bx}{sl}
\newcommand{\tens}[1]{\bm{\mathsfit{#1}}}

% Bold vectors.
\renewcommand{\vec}[1]{\boldsymbol{#1}}

%%%%%%%%%%%%%%%%%%%%%%%%%%%%%%%%%%%%%%%%%%%%%%%%%%%%%%%%%%%%%%%%%%%%%%%%%%%%%%%
%                               Input encoding                                %
%%%%%%%%%%%%%%%%%%%%%%%%%%%%%%%%%%%%%%%%%%%%%%%%%%%%%%%%%%%%%%%%%%%%%%%%%%%%%%%

\usepackage[T1]{fontenc}
\usepackage[utf8]{inputenc}

%%%%%%%%%%%%%%%%%%%%%%%%%%%%%%%%%%%%%%%%%%%%%%%%%%%%%%%%%%%%%%%%%%%%%%%%%%%%%%%
%                         Hyperrefs and PDF metadata                          %
%%%%%%%%%%%%%%%%%%%%%%%%%%%%%%%%%%%%%%%%%%%%%%%%%%%%%%%%%%%%%%%%%%%%%%%%%%%%%%%

\usepackage{hyperref}
\usepackage{lastpage}

% This sets the author in the properties of the PDF as well. If you want to
% change it, just override it with another ``\hypersetup`` call.
\hypersetup{
	breaklinks=false,
	citecolor=darkgreen,
	colorlinks=true,
	linkcolor=darkblue,
	menucolor=black,
	pdfauthor={Martin Ueding},
	urlcolor=darkblue,
}

%%%%%%%%%%%%%%%%%%%%%%%%%%%%%%%%%%%%%%%%%%%%%%%%%%%%%%%%%%%%%%%%%%%%%%%%%%%%%%%
%                               Math Operators                                %
%%%%%%%%%%%%%%%%%%%%%%%%%%%%%%%%%%%%%%%%%%%%%%%%%%%%%%%%%%%%%%%%%%%%%%%%%%%%%%%

% AMS environments like ``align`` and theorems like ``proof``.
\usepackage{amsmath}
\usepackage{amsthm}

% Common math constructs like partial derivatives.
\usepackage{commath}

% Physical units.
\usepackage[output-decimal-marker={,}]{siunitx}

% Since I use mathdesign with italic uppercase greek characters, the Ohm unit will be displayed with an italic Ω by default. Units have to be roman, so this forces it the right way.
\DeclareSIUnit{\ohm}{$\Omegaup$}
\DeclareSIUnit{\division}{DIV}
\DeclareSIUnit{\voltss}{$\mathrm{V_{SS}}$}

% Word like operators.
\DeclareMathOperator{\acosh}{arcosh}
\DeclareMathOperator{\arcosh}{arcosh}
\DeclareMathOperator{\arcsinh}{arsinh}
\DeclareMathOperator{\arsinh}{arsinh}
\DeclareMathOperator{\asinh}{arsinh}
\DeclareMathOperator{\card}{card}
\DeclareMathOperator{\csch}{cshs}
\DeclareMathOperator{\diam}{diam}
\DeclareMathOperator{\sech}{sech}
\renewcommand{\Im}{\mathop{{}\mathrm{Im}}\nolimits}
\renewcommand{\Re}{\mathop{{}\mathrm{Re}}\nolimits}

% Fourier transform.
\DeclareMathOperator{\fourier}{\ensuremath{\mathcal{F}}}

% Roman versions of “e” and “i” to serve as Euler's number and the imaginary
% constant.
\newcommand{\ee}{\eup}
\newcommand{\eup}{\mathrm e}
\newcommand{\ii}{\iup}
\newcommand{\iup}{\mathrm i}

% Symbols for the various mathematical fields (natural numbers, integers,
% rational numbers, real numbers, complex numbers).
\newcommand{\C}{\ensuremath{\mathbb C}}
\newcommand{\N}{\ensuremath{\mathbb N}}
\newcommand{\Q}{\ensuremath{\mathbb Q}}
\newcommand{\R}{\ensuremath{\mathbb R}}
\newcommand{\Z}{\ensuremath{\mathbb Z}}

% Shape like operators.
\DeclareMathOperator{\dalambert}{\Box}
\DeclareMathOperator{\laplace}{\bigtriangleup}
\newcommand{\curl}{\vnabla \times}
\newcommand{\divergence}[1]{\inner{\vnabla}{#1}}
\newcommand{\vnabla}{\vec \nabla}

\newcommand{\half}{\frac 12}

% Unit vector (German „Einheitsvektor“).
\newcommand{\ev}{\hat{\vec e}}

% Scientific notation for large numbers.
\newcommand{\e}[1]{\cdot 10^{#1}}

% Mathematician's notation for the inner (scalar, dot) product.
\newcommand{\bracket}[1]{\left\langle #1 \right\rangle}
\newcommand{\inner}[2]{\bracket{#1, #2}}

% Placeholders.
\newcommand{\emesswert}{\del{\messwert \pm \messwert}}
\newcommand{\fehlt}{\textcolor{darkred}{Hier fehlen noch Inhalte.}}
\newcommand{\messwert}{\textcolor{blue}{\square}}
\newcommand{\punkte}{\phantom{xxxxx}}
\newcommand{\punktevon}[1]{\begin{flushright}/ #1\end{flushright}}

% Separator for equations on a single line.
\newcommand{\eqnsep}{,\quad}

% Quantum Mechanics
\usepackage{braket}

%%%%%%%%%%%%%%%%%%%%%%%%%%%%%%%%%%%%%%%%%%%%%%%%%%%%%%%%%%%%%%%%%%%%%%%%%%%%%%%
%                                  Headings                                   %
%%%%%%%%%%%%%%%%%%%%%%%%%%%%%%%%%%%%%%%%%%%%%%%%%%%%%%%%%%%%%%%%%%%%%%%%%%%%%%%

% This will set fancy headings to the top of the page. The page number will be
% accompanied by the total number of pages. That way, you will know if any page
% is missing.
%
% If you do not want this for your document, you can just use
% ``\pagestyle{plain}``.

\usepackage{scrpage2}

\pagestyle{scrheadings}
\automark{section}
\cfoot{\footnotesize{Seite \thepage\ / \pageref{LastPage}}}
\chead{}
\ihead{}
\ohead{\rightmark}
\setheadsepline{.4pt}

%%%%%%%%%%%%%%%%%%%%%%%%%%%%%%%%%%%%%%%%%%%%%%%%%%%%%%%%%%%%%%%%%%%%%%%%%%%%%%%
%                            Bibliography (BibTeX)                            %
%%%%%%%%%%%%%%%%%%%%%%%%%%%%%%%%%%%%%%%%%%%%%%%%%%%%%%%%%%%%%%%%%%%%%%%%%%%%%%%

\newcommand{\bibliographyfile}{../../central-bibtex/Central}
\bibliographystyle{apalike2}

%%%%%%%%%%%%%%%%%%%%%%%%%%%%%%%%%%%%%%%%%%%%%%%%%%%%%%%%%%%%%%%%%%%%%%%%%%%%%%%
%                                Abbreviations                                %
%%%%%%%%%%%%%%%%%%%%%%%%%%%%%%%%%%%%%%%%%%%%%%%%%%%%%%%%%%%%%%%%%%%%%%%%%%%%%%%

\newcommand{\dhabk}{\mbox{d.\,h.}}

%%%%%%%%%%%%%%%%%%%%%%%%%%%%%%%%%%%%%%%%%%%%%%%%%%%%%%%%%%%%%%%%%%%%%%%%%%%%%%%
%                                  Licences                                   %
%%%%%%%%%%%%%%%%%%%%%%%%%%%%%%%%%%%%%%%%%%%%%%%%%%%%%%%%%%%%%%%%%%%%%%%%%%%%%%%

\usepackage{ccicons}

\newcommand{\ccbysadetext}{%
	\begin{small}
		Dieses Werk bzw. Inhalt steht unter einer
		\href{http://creativecommons.org/licenses/by-sa/3.0/deed.de}{%
			Creative Commons Namensnennung - Weitergabe unter gleichen
		Bedingungen 3.0 Unported Lizenz}.
	\end{small}
}

\newcommand{\ccbysadetitle}{%
	Lizenz: \href{http://creativecommons.org/licenses/by-sa/3.0/deed.de}
	{CC-BY-SA 3.0 \ccbysa}
}
`` in your main
% document and start with ``\title`` and ``\begin{document}`` afterwards.

% If you need to add additional packages, I recommend not doing this in this
% file, but in your main document. That way, you can just drop in a new
% ``header.tex`` and get all the new commands without having to merge manually.

% Since this file encorporates a CC-BY-SA fragment, this whole files is
% licensed under the CC-BY-SA license.

\documentclass[11pt, ngerman, fleqn, DIV=15, BCOR=2cm, headinclude]{scrartcl}

\usepackage{graphicx}

% Environment to quote the problem. Currently, this is just a new name for the
% quote environment.
\newenvironment{problem}{\begin{quote}\textsf{\textbf{Aufgabenstellung:}}\quad}{\end{quote}}

\setkomafont{caption}{\sf}
\setkomafont{captionlabel}{\usekomafont{caption}}

%%%%%%%%%%%%%%%%%%%%%%%%%%%%%%%%%%%%%%%%%%%%%%%%%%%%%%%%%%%%%%%%%%%%%%%%%%%%%%%
%                                Locale, date                                 %
%%%%%%%%%%%%%%%%%%%%%%%%%%%%%%%%%%%%%%%%%%%%%%%%%%%%%%%%%%%%%%%%%%%%%%%%%%%%%%%

\usepackage{babel}
\usepackage[iso]{isodate}

%%%%%%%%%%%%%%%%%%%%%%%%%%%%%%%%%%%%%%%%%%%%%%%%%%%%%%%%%%%%%%%%%%%%%%%%%%%%%%%
%                          Margins and other spacing                          %
%%%%%%%%%%%%%%%%%%%%%%%%%%%%%%%%%%%%%%%%%%%%%%%%%%%%%%%%%%%%%%%%%%%%%%%%%%%%%%%

\usepackage[parfill]{parskip}
\usepackage{setspace}
\usepackage[activate]{microtype}

\setlength{\columnsep}{2cm}

%%%%%%%%%%%%%%%%%%%%%%%%%%%%%%%%%%%%%%%%%%%%%%%%%%%%%%%%%%%%%%%%%%%%%%%%%%%%%%%
%                                    Color                                    %
%%%%%%%%%%%%%%%%%%%%%%%%%%%%%%%%%%%%%%%%%%%%%%%%%%%%%%%%%%%%%%%%%%%%%%%%%%%%%%%

\usepackage[usenames, dvipsnames]{xcolor}

\colorlet{darkred}{red!70!black}
\colorlet{darkblue}{blue!70!black}
\colorlet{darkgreen}{green!40!black}

%%%%%%%%%%%%%%%%%%%%%%%%%%%%%%%%%%%%%%%%%%%%%%%%%%%%%%%%%%%%%%%%%%%%%%%%%%%%%%%
%                         Font and font like settings                         %
%%%%%%%%%%%%%%%%%%%%%%%%%%%%%%%%%%%%%%%%%%%%%%%%%%%%%%%%%%%%%%%%%%%%%%%%%%%%%%%

% This replaces all fonts with Bitstream Charter, Bitstream Vera Sans and
% Bitstream Vera Mono. Math will be rendered in Charter.
\usepackage[charter, greekuppercase=italicized]{mathdesign}
\usepackage{beramono}
\usepackage{berasans}

% Bold, sans-serif tensors. This fragment is taken from “egreg” from
% http://tex.stackexchange.com/a/82747/8945 and licensed under `CC-BY-SA
% <https://creativecommons.org/licenses/by-sa/3.0/>`_.
\usepackage{bm}
\DeclareMathAlphabet{\mathsfit}{\encodingdefault}{\sfdefault}{m}{sl}
\SetMathAlphabet{\mathsfit}{bold}{\encodingdefault}{\sfdefault}{bx}{sl}
\newcommand{\tens}[1]{\bm{\mathsfit{#1}}}

% Bold vectors.
\renewcommand{\vec}[1]{\boldsymbol{#1}}

%%%%%%%%%%%%%%%%%%%%%%%%%%%%%%%%%%%%%%%%%%%%%%%%%%%%%%%%%%%%%%%%%%%%%%%%%%%%%%%
%                               Input encoding                                %
%%%%%%%%%%%%%%%%%%%%%%%%%%%%%%%%%%%%%%%%%%%%%%%%%%%%%%%%%%%%%%%%%%%%%%%%%%%%%%%

\usepackage[T1]{fontenc}
\usepackage[utf8]{inputenc}

%%%%%%%%%%%%%%%%%%%%%%%%%%%%%%%%%%%%%%%%%%%%%%%%%%%%%%%%%%%%%%%%%%%%%%%%%%%%%%%
%                         Hyperrefs and PDF metadata                          %
%%%%%%%%%%%%%%%%%%%%%%%%%%%%%%%%%%%%%%%%%%%%%%%%%%%%%%%%%%%%%%%%%%%%%%%%%%%%%%%

\usepackage{hyperref}
\usepackage{lastpage}

% This sets the author in the properties of the PDF as well. If you want to
% change it, just override it with another ``\hypersetup`` call.
\hypersetup{
	breaklinks=false,
	citecolor=darkgreen,
	colorlinks=true,
	linkcolor=darkblue,
	menucolor=black,
	pdfauthor={Martin Ueding},
	urlcolor=darkblue,
}

%%%%%%%%%%%%%%%%%%%%%%%%%%%%%%%%%%%%%%%%%%%%%%%%%%%%%%%%%%%%%%%%%%%%%%%%%%%%%%%
%                               Math Operators                                %
%%%%%%%%%%%%%%%%%%%%%%%%%%%%%%%%%%%%%%%%%%%%%%%%%%%%%%%%%%%%%%%%%%%%%%%%%%%%%%%

% AMS environments like ``align`` and theorems like ``proof``.
\usepackage{amsmath}
\usepackage{amsthm}

% Common math constructs like partial derivatives.
\usepackage{commath}

% Physical units.
\usepackage[output-decimal-marker={,}]{siunitx}

% Since I use mathdesign with italic uppercase greek characters, the Ohm unit will be displayed with an italic Ω by default. Units have to be roman, so this forces it the right way.
\DeclareSIUnit{\ohm}{$\Omegaup$}
\DeclareSIUnit{\division}{DIV}
\DeclareSIUnit{\voltss}{$\mathrm{V_{SS}}$}

% Word like operators.
\DeclareMathOperator{\acosh}{arcosh}
\DeclareMathOperator{\arcosh}{arcosh}
\DeclareMathOperator{\arcsinh}{arsinh}
\DeclareMathOperator{\arsinh}{arsinh}
\DeclareMathOperator{\asinh}{arsinh}
\DeclareMathOperator{\card}{card}
\DeclareMathOperator{\csch}{cshs}
\DeclareMathOperator{\diam}{diam}
\DeclareMathOperator{\sech}{sech}
\renewcommand{\Im}{\mathop{{}\mathrm{Im}}\nolimits}
\renewcommand{\Re}{\mathop{{}\mathrm{Re}}\nolimits}

% Fourier transform.
\DeclareMathOperator{\fourier}{\ensuremath{\mathcal{F}}}

% Roman versions of “e” and “i” to serve as Euler's number and the imaginary
% constant.
\newcommand{\ee}{\eup}
\newcommand{\eup}{\mathrm e}
\newcommand{\ii}{\iup}
\newcommand{\iup}{\mathrm i}

% Symbols for the various mathematical fields (natural numbers, integers,
% rational numbers, real numbers, complex numbers).
\newcommand{\C}{\ensuremath{\mathbb C}}
\newcommand{\N}{\ensuremath{\mathbb N}}
\newcommand{\Q}{\ensuremath{\mathbb Q}}
\newcommand{\R}{\ensuremath{\mathbb R}}
\newcommand{\Z}{\ensuremath{\mathbb Z}}

% Shape like operators.
\DeclareMathOperator{\dalambert}{\Box}
\DeclareMathOperator{\laplace}{\bigtriangleup}
\newcommand{\curl}{\vnabla \times}
\newcommand{\divergence}[1]{\inner{\vnabla}{#1}}
\newcommand{\vnabla}{\vec \nabla}

\newcommand{\half}{\frac 12}

% Unit vector (German „Einheitsvektor“).
\newcommand{\ev}{\hat{\vec e}}

% Scientific notation for large numbers.
\newcommand{\e}[1]{\cdot 10^{#1}}

% Mathematician's notation for the inner (scalar, dot) product.
\newcommand{\bracket}[1]{\left\langle #1 \right\rangle}
\newcommand{\inner}[2]{\bracket{#1, #2}}

% Placeholders.
\newcommand{\emesswert}{\del{\messwert \pm \messwert}}
\newcommand{\fehlt}{\textcolor{darkred}{Hier fehlen noch Inhalte.}}
\newcommand{\messwert}{\textcolor{blue}{\square}}
\newcommand{\punkte}{\phantom{xxxxx}}
\newcommand{\punktevon}[1]{\begin{flushright}/ #1\end{flushright}}

% Separator for equations on a single line.
\newcommand{\eqnsep}{,\quad}

% Quantum Mechanics
\usepackage{braket}

%%%%%%%%%%%%%%%%%%%%%%%%%%%%%%%%%%%%%%%%%%%%%%%%%%%%%%%%%%%%%%%%%%%%%%%%%%%%%%%
%                                  Headings                                   %
%%%%%%%%%%%%%%%%%%%%%%%%%%%%%%%%%%%%%%%%%%%%%%%%%%%%%%%%%%%%%%%%%%%%%%%%%%%%%%%

% This will set fancy headings to the top of the page. The page number will be
% accompanied by the total number of pages. That way, you will know if any page
% is missing.
%
% If you do not want this for your document, you can just use
% ``\pagestyle{plain}``.

\usepackage{scrpage2}

\pagestyle{scrheadings}
\automark{section}
\cfoot{\footnotesize{Seite \thepage\ / \pageref{LastPage}}}
\chead{}
\ihead{}
\ohead{\rightmark}
\setheadsepline{.4pt}

%%%%%%%%%%%%%%%%%%%%%%%%%%%%%%%%%%%%%%%%%%%%%%%%%%%%%%%%%%%%%%%%%%%%%%%%%%%%%%%
%                            Bibliography (BibTeX)                            %
%%%%%%%%%%%%%%%%%%%%%%%%%%%%%%%%%%%%%%%%%%%%%%%%%%%%%%%%%%%%%%%%%%%%%%%%%%%%%%%

\newcommand{\bibliographyfile}{../../central-bibtex/Central}
\bibliographystyle{apalike2}

%%%%%%%%%%%%%%%%%%%%%%%%%%%%%%%%%%%%%%%%%%%%%%%%%%%%%%%%%%%%%%%%%%%%%%%%%%%%%%%
%                                Abbreviations                                %
%%%%%%%%%%%%%%%%%%%%%%%%%%%%%%%%%%%%%%%%%%%%%%%%%%%%%%%%%%%%%%%%%%%%%%%%%%%%%%%

\newcommand{\dhabk}{\mbox{d.\,h.}}

%%%%%%%%%%%%%%%%%%%%%%%%%%%%%%%%%%%%%%%%%%%%%%%%%%%%%%%%%%%%%%%%%%%%%%%%%%%%%%%
%                                  Licences                                   %
%%%%%%%%%%%%%%%%%%%%%%%%%%%%%%%%%%%%%%%%%%%%%%%%%%%%%%%%%%%%%%%%%%%%%%%%%%%%%%%

\usepackage{ccicons}

\newcommand{\ccbysadetext}{%
	\begin{small}
		Dieses Werk bzw. Inhalt steht unter einer
		\href{http://creativecommons.org/licenses/by-sa/3.0/deed.de}{%
			Creative Commons Namensnennung - Weitergabe unter gleichen
		Bedingungen 3.0 Unported Lizenz}.
	\end{small}
}

\newcommand{\ccbysadetitle}{%
	Lizenz: \href{http://creativecommons.org/licenses/by-sa/3.0/deed.de}
	{CC-BY-SA 3.0 \ccbysa}
}


\usepackage{keystroke}
\usepackage{placeins}
\usepackage{pdflscape}
\usepackage{booktabs}
\usepackage{xfrac}
\usepackage{minted}
\usepackage{multicol}

\newcommand\versuchsnummer{443}

\ihead{physik412 – Versuch \versuchsnummer}
\ifoot{Martin Ueding, Lino Lemmer}

\subject{Praktikumsprotokoll}
\title{Kernmagnetische Relaxation}
\subtitle{physik412 – Versuch \versuchsnummer}
\author{
    Martin Ueding \\
    \small{\href{mailto:mu@martin-ueding.de}{mu@martin-ueding.de}}
    \and
    Lino Lemmer \\
    \small{\href{mailto:s6lilemm@uni-bonn.de}{s6lilemm@uni-bonn.de}}
}

\setcounter{secnumdepth}{4}
\setcounter{tocdepth}{4}

\begin{document}

\maketitle

\begin{abstract}
    \fehlt
\end{abstract}

\tableofcontents

%%%%%%%%%%%%%%%%%%%%%%%%%%%%%%%%%%%%%%%%%%%%%%%%%%%%%%%%%%%%%%%%%%%%%%%%%%%%%%%
%                                   Theorie                                   %
%%%%%%%%%%%%%%%%%%%%%%%%%%%%%%%%%%%%%%%%%%%%%%%%%%%%%%%%%%%%%%%%%%%%%%%%%%%%%%%

\chapter{Theorie}

\cite{teach_spin_manual}

\cite[Abschnitt~15.9.3 „Magnetische~Resonanz”]{meschede-gerthsen_24}

\cite{physik412-Anleitung}

\section{Spin $\sfrac12$ Teilchen im homogenen Magnetfeld}

In diesem Versuch untersuchen wie die magnetische Resonanz von Protonen, die in
Form von Wasserstoff in einer großen Form in der Probe aus leichtem Mineralöl
vorliegen. Die Protonen zeichnen sich durch einen Spin $\sfrac12$ aus.

Dieser Spin wurde in „Experimentalphysik 4“ eingeführt und verhält sich wie ein
Drehimpuls. Die Quantenzahl $s$ ist hier immer $\sfrac12$. Es bleibt die
$z$-Komponente $m_s$, die die Werte $-\sfrac12$ und $\sfrac12$ annehmen kann.

Die Protonen besitzen ein magnetisches Moment $\vec\mu$, das immer parallel zum
Spin ist und eine feste Magnitude besitzt:
\[
    \mu = \gamma \hbar s
\]

Der Faktor $\gamma$ ist das gyromagnetische Verhältnis und ist für verschiedene
Teilchen (z.\,B. Protonen, Elektronen) unterschiedlich.

\subsection{Aufspaltung in zwei Zustände}

In einem homogenen Magnetfeld $\vec B_0$ erfährt das magnetische Moment
$\vec\mu$ ein Drehmoment $\vec M = \vec \mu \times \vec B_0$, das zu einer
Einstellenergie führt:
\[
    E = - \inner{\vec \mu}{\vec B_0}
\]

Da die magnetische Spinquantenzahl $m_s$ nur die Werte $-\sfrac12$ und $\sfrac12$ annehmen
darf, gibt es nur zwei Zustände, $\ket\uparrow$ und $\ket\downarrow$, die
aufgrund der Einstellenergie unterschiedliche Energie haben. Ohne das externe
Magnetfeld wären diese Zustände entartet.

\subsection{Besetzung, thermisches Gleichgewicht}

Ohne das externe Feld liegen alle Spins in einem beliebigen Zustand vor, so
dass es keine makroskopische Magnetisierung $\vec M$ gibt.

\subsection{Longitudinale Relaxation (Relaxationszeit $T_1$)}

Ist das System in einem beliebigen Zustand, so werden die Spins nach einiger
Zeit in den Grundzustand wechseln. Die Rate, mit der die Spins wechseln ist
proportional zur Anzahl, die noch nicht gewechselt sind. Aus diesem Ansatz
lässt sich mit geeigneten Anfangsbedingungen folgende Zeitentwicklung
herleiten:
\begin{equation}
    \label{eq:}
    \vec M(t) = \vec M_0 \del{1 - \exp\del{- \frac{t}{T_1}}}
\end{equation}

Dabei haben wir die Zeitkonstante $T_1$ eingeführt, die beschreibt, wie schnell
diese Relaxation vor sich geht. Diese Zeit ist relativ lang.

\subsection{Präzession um $z$–Achse}
\label{subsec:Präzession}

Da das magnetische Moment $\vec\mu$ wahrscheinlich nicht parallel zum
magnetischen Feld ist, wird auf es konstant das Drehmoment durch das
magnetische Feld wirken. Dies führt dazu, dass das magnetische Moment um die
$z$-Achse präzessiert. Die Frequenz dieser Präzession ist die Larmorfrequenz:
\[
    \omega_\text L = \gamma B_0
\]

Die makroskopische Magnetisierung, die sich als Mittel aus allen magnetischen
Momenten ergibt, wird daher auch um die $z$-Achse präzessieren, zumindest
solange die einzelnen Spins mit der gleichen Frequenz präzessieren. Dazu später
mehr.

\section{Reaktion von Spin auf zusätzliches Feld}

In diesem Experiment werden wir zusätzlich zum (möglichst) homogenen Feld $\vec
B_0$ ein gepulstes, sich drehendes, Feld $B_\text{RF}$ verwenden, dessen
Frequenz im Radiobereich (\si{\mega\hertz}) liegt.

\subsection{Blochkugel}

Die Spins haben im externen Feld die Basiszustände $\ket\uparrow$ und
$\ket\downarrow$. Es können aber auch Zustände auftreten, die sich durch zwei
Mischwinkel $\theta$ und $\phi$ charakterisieren lassen:
\parencite{wikipedia/bloch_kugel}
\[
    \ket{\theta, \phi}
    = \cos\del{\frac\theta2} \ket\uparrow
    + \eup^{\iup \phi} \sin\del{\frac\theta2} \ket\downarrow
\]

Durch diese Superposition ist es möglich, dass das magnetische Moment in jede
beliebige Richtung liegen kann. Die energetisch günstigste Variante ist jedoch,
wenn der Spin möglichst parallel zum Magnetfeld $\vec B_0$ liegt, also
$\ket\uparrow$.

Im weiteren Verlauf können wir uns die Orientierung des Spins als
Einheitsvektor in der Kugel vorstellen.

\subsection{Rotierendes Bezugssystem}

Wie schon in \ref{subsec:Präzession} beschrieben, präzessiert das magnetische
Moment und der Spin um die $z$-Achse um ruhenden Inertialsystem des Labors
$\Sigma$. Es ist für die weiteren Überlegungen
hilfreich, ein rotierendes Bezugssystem $\Sigma^*$ zu betrachten, in dem das
magnetische Moment fest ist.

\subsection{$\piup/2$–Puls}

Ein Puls des rotierenden Magnetfeldes der richtigen Stärke und länge wird dazu
führen, dass das magnetische Moment für den Augenblick nur um die mitgeführte
$x$-Achse, also die $x^*$-Achse präzessiert und somit von der $z^*$ auf die
$y^*$-Achse kippt. Ist dies erreicht, muss der Puls beendet sein, damit das
Moment nicht noch weiter kippt.

Das magnetische Moment einer überwiegenden Anzahl Teilchen liegt nun auf der
$x^*$-Achse. Somit liegt auch die Magnetisierung $\vec M$ auf der $x^*$-Achse.
Da sich das Bezugssystem um die $z$-Achse dreht, ist auf der (festen) $x$-Achse
nun eine makroskopische Magnetisierung zu beobachten, die sich zeitlich
sinusförmig verhält.

Mit der Messspule ist nun die mit der Larmorfrequenz oszillierende
Magnetisierung zu beobachten.

An dieser Stelle ist \cite[Abbildung~15.55 auf Seite~765]{meschede-gerthsen_24}
recht illustrativ.

\subsection{$\piup$–Puls}

Ähnlich wie der $\piup/2$-Puls kippt der $\piup$-Puls das magnetische Moment,
jedoch um den doppelten Winkel. Im thermalisierten Grundzustand des Systems
invertiert dieser Puls $\ket\uparrow$ und $\ket\downarrow$. Geht dem
$\piup$-Puls ein $\piup/2$-Puls voraus, so wird der Azimutwinkel $\phi$ des
Zustandes verändert.

\subsection{Transversale Relaxation}

Die Larmorfrequenz an jeder Stelle fast gleich, jedoch nur fast. Dadurch
präzessieren die einzelnen Spins mit einer lokal leicht verschiedenen
Geschwindigkeit. Auch wenn die Unterschiede nur klein sind, reicht es aus, wenn
die relative Verschiebung zwischen den Spins $\piup$ beträgt, dass keine
makroskopische transversale Magnetisierung gemessen werden kann.

Zum einen wird die Ortsabhängigkeit der Larmorfrequenz durch das leicht
inhomogene Magnetfeld des großen äußeren Magneten verursacht. Diese Veränderung
ist konstant und kann, wie wir später beschreiben werden, mit der
Hahn-Spinecho-Sequenz rückgängig gemacht werden.

Außerdem erzeugen die magnetischen Momente der einzelnen Spins für sich auch
wieder Magnetfelder $\vec H$, die auch zu einer Verschiebung der Larmorfrequenz
führen. Dieser Effekt ist stark zeitabhängig und kann nicht korrigiert werden.

Da es diese verschiedenen Effekte gibt, werden zwei verschiedene
Relaxationszeiten eingeführt.

\subsubsection{Homogene Transversale Relaxationszeit $T_2$}

Durch einen $\piup/2$-Puls wird die makroskopische Magnetisierung in die
$x$-$y$-Ebene gekippt und präzessiert dann mit der Larmorfrequenz um die
$z$-Achse, jedoch zerläuft diese Magnetisierung aufgrund der leicht
verschiedenen Larmorfrequenzen. Dies nennt man „freien Induktionszerfall“, kurz
„FID“\footnote{„FID“ kommt von „free induction decay”.}.

Wäre das Magnetfeld perfekt homogen, so basierte die Frequenzverschiebung nur
auf dem Magnetfeld der anderen Spins. Diese homogene Relaxation ist die
eigentlich interessante Zeitkonstante, da sie vom nur Material abhängt und
nicht vom experimentellen Aufbau.

Mit der Hahn-Spinecho-Sequenz können wir später diese Zeit unabhängig vom
externen Magnetfeld messen.

Hier ist ein exponentieller Abfall der Gesamtmagnetisierung zu erwarten, die
wir mit der Zeitkonstante $T_2$ beschreiben werden.

\subsubsection{Effektive Transversale Relaxationszeit $T_2^*$}

Das inhomogene Magnetfeld verstärkt den freien Induktionszerfall. Auch hier ist
ein exponentieller Abfall zu beobachten, der jedoch schneller ist, als der
Effekt durch die homogene Relaxation alleine. Daher ist $T_2^*$ kürzer als
$T_2$.

Mit der inhomogenen transversalen Relaxationszeit $T_{2,\text{inhom}}$ hängen
die anderen wie folgt zusammen: \parencite[Formel~P443.5]{physik412-Anleitung}
\begin{equation}
    \label{eq:}
    \frac{1}{T_2^*} = \frac 1{T_2} + \frac{1}{T_{2,\text{inhom}}}
\end{equation}

\subsection{Rabi-Oszillation}

Rabi-Oszillationen sind allgemein Oszillationen eines Zweizustandsystems durch
eine externe Anregung. \cite{wikipedia/Rabi_Oszillation}

Hier ist diese externe Anregung der RF-Puls, den wir durch die Spule schicken.
Dabei legen wir die Frequenz des Pulses möglichst genau auf die Larmorfrequenz,
und variieren hier die Pulslänge. Wir erwarten, dass sich bei einem sehr kurzen
Puls die makroskopische Magnetisierung nicht verändert. Wenn der Puls länger
wird, und in den Bereich des $\pi/2$-Puls kommt, ist ein maximales Signal zu
erwarten. Wird der Puls länger, so dass er in der Nähe des $\piup$-Pulses
kommt, ist wieder kein Signal zu erwarten. In der Größenordnung eines
$3\piup/4$-Pulses sollte die Amplitude wieder maximal sein, jedoch die Phase
umgekehrt sein.

\cite[Abschnitt~15.9.5 „Rabi-Atomstrahlresonanz”]{meschede-gerthsen_24}

\section{Messmethoden}

\subsection{Konstante Grundspannung}

Wir betreiben das Oszilloskop mit DC-Kopplung, so dass allen Messung ein
konstante Grundspannung zugrundeliegen kann, die wir bei jeder Messung noch
abziehen müssen.

\subsection{Longitudinale Relaxationszeit $T_1$}
\subsubsection{Sättigungs–Zurückgewinnung}

Formel P443.1

Formel P443.2

\subsubsection{Polarisations–Zurückgewinnung}

Formel P443.3

Formel P443.4

\subsection{Transversale Relaxationszeit $T_2$}
\subsubsection{FID-Signal}
\subsubsection{Hahn–Spinecho–Sequenz}
\subsubsection{Carr–Purcell–Sequenz}
\subsubsection{Meiboom–Gill–Sequenz}

\chapter{Aufbau und Kalibrierung}
\section{Magnet und PS2 Controller}
\section{Mainframe und Oszilloskop}

\section{Temperatureinstellung}
\section{Tuning des RF-Resonanzkreises}

Wir stellen \texttt{A\_len} auf \SI{2.5}{\micro\second} und \texttt P auf
\SI{100}{\micro\second}.

\texttt{print\_001}

\texttt{print\_002.csv} zeigt die Antwort bei 10 ns/div

\SI{47.4}{\nano\second}

\section{Optimierung des Free Induction Decay Signals}

Wir stellen das Oszilloskop so ein, dass wir das Signal gut sehen können.

Die Gradienten haben wir eingestellt auf: $X = 0.5$, $Y = 18.0$, $Z = 32.2$ und
$Z^2 = 0.0$.

21.16141

003

\section{$\frac\pi2$- und $\pi$-Puls Justage}

Die Pulsdauer, bei der das Signal minimal ist, erhalten wir bei \texttt{A\_len}
= \SI{5.16}{\micro\second}. Die Dauer für das $\piup/2$-Signal ist entsprechend
\SI{2.58}{\micro\second}.

\chapter{Durchführung und Auswertung}

\FloatBarrier
\section{Rabi-Oszillationen}

Unsere Messwerte befinden sich in Tabelle~\ref{tab:rabi}, eine graphische
Darstellung einschließlich Anpassungsfunktionen ist in Abbildung~\ref{fig:rabi}
zu sehen.

\begin{table}
    \centering
    \begin{tabular}{SSSS}
        {$f / \si{\mega\hertz}$} &
        {A\_len$ / \si{\micro\second}$} &
        {Signal Gelb$ / \si{\volt}$} &
        {Signal Grün$ / \si{\volt}$} \\
        \midrule
        %< for f, A_len, gelb_val, gruen_val in rabi_messdaten: ->%
        << f >> & << A_len >> & << gelb_val >> & << gruen_val >> \\
        %< endfor >%
    \end{tabular}
    \caption{%
        Messwerte zur Rabi-Oszillation
    }
    \label{tab:rabi}
\end{table}

\begin{figure}[htbp]
    \centering
    \includegraphics[width=\linewidth]{Rabi.pdf}
    \caption{%
        Messdaten und angepasste Funktionen zur Rabi-Oszillation.
    }
    \label{fig:rabi}
\end{figure}

\FloatBarrier
\section{Longitudinale Relaxationszeit $T_1$}


$\nu = \SI{21.1573}{\mega\hertz}$

$\texttt{A\_test} = \SI{5.56}{\micro\second}$

\FloatBarrier
\subsection{Sättigungs-Zurückgewinnung}

\[
    M_0 \hat = \SI{7.525}{\volt}
\]

\begin{figure}[htbp]
    \centering
    \includegraphics[width=\linewidth]{Saettigung.pdf}
    \caption{%
    }
    \label{fig:}
\end{figure}

\FloatBarrier
\subsection{Polarisations-Zurückgewinnung}

\begin{figure}[htbp]
    \centering
    \includegraphics[width=\linewidth]{Polarisation.pdf}
    \caption{%
    }
    \label{fig:}
\end{figure}

\FloatBarrier
\section{Effektive Transversale Relaxationszeit $T_2^*$}

\begin{figure}[htbp]
    \centering
    \includegraphics[width=\linewidth]{effektiv.pdf}
    \caption{%
    }
    \label{fig:}
\end{figure}

\FloatBarrier
\subsection{Homogene Transversale Relaxationszeit $T_2$}

\FloatBarrier
\subsubsection{Hahn–Spinecho–Sequenz}

\begin{figure}[htbp]
    \centering
    \includegraphics[width=\linewidth]{Hahn.pdf}
    \caption{%
    }
    \label{fig:}
\end{figure}

\FloatBarrier
\subsubsection{Carr–Purcell–Sequenz}

$\tau = \SI{5.00}{\micro\second}$
$\texttt{A\_len} = \SI{2.66}{\micro\second}$
$\texttt{B\_len} = \SI{5.32}{\micro\second}$
$\texttt{P} = \SI{580}{\milli\second}$
$\texttt{N} = \num{20}$

\FloatBarrier
\subsubsection{Meiboom–Gill–Sequenz}

%%%%%%%%%%%%%%%%%%%%%%%%%%%%%%%%%%%%%%%%%%%%%%%%%%%%%%%%%%%%%%%%%%%%%%%%%%%%%%%
%                               Zusammenfassung                               %
%%%%%%%%%%%%%%%%%%%%%%%%%%%%%%%%%%%%%%%%%%%%%%%%%%%%%%%%%%%%%%%%%%%%%%%%%%%%%%%

\FloatBarrier
\chapter{Zusammenfassung}

\fehlt

%%%%%%%%%%%%%%%%%%%%%%%%%%%%%%%%%%%%%%%%%%%%%%%%%%%%%%%%%%%%%%%%%%%%%%%%%%%%%%%
%                                   Anhang                                    %
%%%%%%%%%%%%%%%%%%%%%%%%%%%%%%%%%%%%%%%%%%%%%%%%%%%%%%%%%%%%%%%%%%%%%%%%%%%%%%%

\FloatBarrier
\begin{appendix}
    \FloatBarrier
    \chapter{\LaTeX-Quelltext}

    Der \LaTeX-Quelltext zu allen Protokollen in diesem Praktikum kann auf
    \ref{it:mu} eingesehen werden. Die Quellen für alle Protokolle in diesem
    Praktikum können auf \ref{it:github/alles} eingesehen werden. Die
    \LaTeX-Datei wird aus \ref{it:github/template} generiert.

    \begin{enumerate}
        \item
            \label{it:mu}
            \url{http://martin-ueding.de/de/university/physik412/}
        \item
            \label{it:github/alles}
            \url{https://github.com/martin-ueding/physik412-Protokolle/}
        \item
            \label{it:github/template}
            \url{https://github.com/martin-ueding/physik412-Protokolle/blob/master/\versuchsnummer/Template.tex}
    \end{enumerate}
\end{appendix}

%%%%%%%%%%%%%%%%%%%%%%%%%%%%%%%%%%%%%%%%%%%%%%%%%%%%%%%%%%%%%%%%%%%%%%%%%%%%%%%
%                                  Literatur                                  %
%%%%%%%%%%%%%%%%%%%%%%%%%%%%%%%%%%%%%%%%%%%%%%%%%%%%%%%%%%%%%%%%%%%%%%%%%%%%%%%

\FloatBarrier
\printbibliography

\end{document}

% vim: et spell spelllang=de tw=79
