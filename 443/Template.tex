% Copyright © 2013 Martin Ueding <dev@martin-ueding.de>

% Copyright © 2012-2013 Martin Ueding <dev@martin-ueding.de>

% This is my general purpose LaTeX header file for writing German documents.
% Ideally, you include this using a simple ``% Copyright © 2012-2013 Martin Ueding <dev@martin-ueding.de>

% This is my general purpose LaTeX header file for writing German documents.
% Ideally, you include this using a simple ``% Copyright © 2012-2013 Martin Ueding <dev@martin-ueding.de>

% This is my general purpose LaTeX header file for writing German documents.
% Ideally, you include this using a simple ``\input{header.tex}`` in your main
% document and start with ``\title`` and ``\begin{document}`` afterwards.

% If you need to add additional packages, I recommend not doing this in this
% file, but in your main document. That way, you can just drop in a new
% ``header.tex`` and get all the new commands without having to merge manually.

% Since this file encorporates a CC-BY-SA fragment, this whole files is
% licensed under the CC-BY-SA license.

\documentclass[11pt, ngerman, fleqn, DIV=15, BCOR=2cm, headinclude]{scrartcl}

\usepackage{graphicx}

% Environment to quote the problem. Currently, this is just a new name for the
% quote environment.
\newenvironment{problem}{\begin{quote}\textsf{\textbf{Aufgabenstellung:}}\quad}{\end{quote}}

\setkomafont{caption}{\sf}
\setkomafont{captionlabel}{\usekomafont{caption}}

%%%%%%%%%%%%%%%%%%%%%%%%%%%%%%%%%%%%%%%%%%%%%%%%%%%%%%%%%%%%%%%%%%%%%%%%%%%%%%%
%                                Locale, date                                 %
%%%%%%%%%%%%%%%%%%%%%%%%%%%%%%%%%%%%%%%%%%%%%%%%%%%%%%%%%%%%%%%%%%%%%%%%%%%%%%%

\usepackage{babel}
\usepackage[iso]{isodate}

%%%%%%%%%%%%%%%%%%%%%%%%%%%%%%%%%%%%%%%%%%%%%%%%%%%%%%%%%%%%%%%%%%%%%%%%%%%%%%%
%                          Margins and other spacing                          %
%%%%%%%%%%%%%%%%%%%%%%%%%%%%%%%%%%%%%%%%%%%%%%%%%%%%%%%%%%%%%%%%%%%%%%%%%%%%%%%

\usepackage[parfill]{parskip}
\usepackage{setspace}
\usepackage[activate]{microtype}

\setlength{\columnsep}{2cm}

%%%%%%%%%%%%%%%%%%%%%%%%%%%%%%%%%%%%%%%%%%%%%%%%%%%%%%%%%%%%%%%%%%%%%%%%%%%%%%%
%                                    Color                                    %
%%%%%%%%%%%%%%%%%%%%%%%%%%%%%%%%%%%%%%%%%%%%%%%%%%%%%%%%%%%%%%%%%%%%%%%%%%%%%%%

\usepackage[usenames, dvipsnames]{xcolor}

\colorlet{darkred}{red!70!black}
\colorlet{darkblue}{blue!70!black}
\colorlet{darkgreen}{green!40!black}

%%%%%%%%%%%%%%%%%%%%%%%%%%%%%%%%%%%%%%%%%%%%%%%%%%%%%%%%%%%%%%%%%%%%%%%%%%%%%%%
%                         Font and font like settings                         %
%%%%%%%%%%%%%%%%%%%%%%%%%%%%%%%%%%%%%%%%%%%%%%%%%%%%%%%%%%%%%%%%%%%%%%%%%%%%%%%

% This replaces all fonts with Bitstream Charter, Bitstream Vera Sans and
% Bitstream Vera Mono. Math will be rendered in Charter.
\usepackage[charter, greekuppercase=italicized]{mathdesign}
\usepackage{beramono}
\usepackage{berasans}

% Bold, sans-serif tensors. This fragment is taken from “egreg” from
% http://tex.stackexchange.com/a/82747/8945 and licensed under `CC-BY-SA
% <https://creativecommons.org/licenses/by-sa/3.0/>`_.
\usepackage{bm}
\DeclareMathAlphabet{\mathsfit}{\encodingdefault}{\sfdefault}{m}{sl}
\SetMathAlphabet{\mathsfit}{bold}{\encodingdefault}{\sfdefault}{bx}{sl}
\newcommand{\tens}[1]{\bm{\mathsfit{#1}}}

% Bold vectors.
\renewcommand{\vec}[1]{\boldsymbol{#1}}

%%%%%%%%%%%%%%%%%%%%%%%%%%%%%%%%%%%%%%%%%%%%%%%%%%%%%%%%%%%%%%%%%%%%%%%%%%%%%%%
%                               Input encoding                                %
%%%%%%%%%%%%%%%%%%%%%%%%%%%%%%%%%%%%%%%%%%%%%%%%%%%%%%%%%%%%%%%%%%%%%%%%%%%%%%%

\usepackage[T1]{fontenc}
\usepackage[utf8]{inputenc}

%%%%%%%%%%%%%%%%%%%%%%%%%%%%%%%%%%%%%%%%%%%%%%%%%%%%%%%%%%%%%%%%%%%%%%%%%%%%%%%
%                         Hyperrefs and PDF metadata                          %
%%%%%%%%%%%%%%%%%%%%%%%%%%%%%%%%%%%%%%%%%%%%%%%%%%%%%%%%%%%%%%%%%%%%%%%%%%%%%%%

\usepackage{hyperref}
\usepackage{lastpage}

% This sets the author in the properties of the PDF as well. If you want to
% change it, just override it with another ``\hypersetup`` call.
\hypersetup{
	breaklinks=false,
	citecolor=darkgreen,
	colorlinks=true,
	linkcolor=darkblue,
	menucolor=black,
	pdfauthor={Martin Ueding},
	urlcolor=darkblue,
}

%%%%%%%%%%%%%%%%%%%%%%%%%%%%%%%%%%%%%%%%%%%%%%%%%%%%%%%%%%%%%%%%%%%%%%%%%%%%%%%
%                               Math Operators                                %
%%%%%%%%%%%%%%%%%%%%%%%%%%%%%%%%%%%%%%%%%%%%%%%%%%%%%%%%%%%%%%%%%%%%%%%%%%%%%%%

% AMS environments like ``align`` and theorems like ``proof``.
\usepackage{amsmath}
\usepackage{amsthm}

% Common math constructs like partial derivatives.
\usepackage{commath}

% Physical units.
\usepackage[output-decimal-marker={,}]{siunitx}

% Since I use mathdesign with italic uppercase greek characters, the Ohm unit will be displayed with an italic Ω by default. Units have to be roman, so this forces it the right way.
\DeclareSIUnit{\ohm}{$\Omegaup$}
\DeclareSIUnit{\division}{DIV}
\DeclareSIUnit{\voltss}{$\mathrm{V_{SS}}$}

% Word like operators.
\DeclareMathOperator{\acosh}{arcosh}
\DeclareMathOperator{\arcosh}{arcosh}
\DeclareMathOperator{\arcsinh}{arsinh}
\DeclareMathOperator{\arsinh}{arsinh}
\DeclareMathOperator{\asinh}{arsinh}
\DeclareMathOperator{\card}{card}
\DeclareMathOperator{\csch}{cshs}
\DeclareMathOperator{\diam}{diam}
\DeclareMathOperator{\sech}{sech}
\renewcommand{\Im}{\mathop{{}\mathrm{Im}}\nolimits}
\renewcommand{\Re}{\mathop{{}\mathrm{Re}}\nolimits}

% Fourier transform.
\DeclareMathOperator{\fourier}{\ensuremath{\mathcal{F}}}

% Roman versions of “e” and “i” to serve as Euler's number and the imaginary
% constant.
\newcommand{\ee}{\eup}
\newcommand{\eup}{\mathrm e}
\newcommand{\ii}{\iup}
\newcommand{\iup}{\mathrm i}

% Symbols for the various mathematical fields (natural numbers, integers,
% rational numbers, real numbers, complex numbers).
\newcommand{\C}{\ensuremath{\mathbb C}}
\newcommand{\N}{\ensuremath{\mathbb N}}
\newcommand{\Q}{\ensuremath{\mathbb Q}}
\newcommand{\R}{\ensuremath{\mathbb R}}
\newcommand{\Z}{\ensuremath{\mathbb Z}}

% Shape like operators.
\DeclareMathOperator{\dalambert}{\Box}
\DeclareMathOperator{\laplace}{\bigtriangleup}
\newcommand{\curl}{\vnabla \times}
\newcommand{\divergence}[1]{\inner{\vnabla}{#1}}
\newcommand{\vnabla}{\vec \nabla}

\newcommand{\half}{\frac 12}

% Unit vector (German „Einheitsvektor“).
\newcommand{\ev}{\hat{\vec e}}

% Scientific notation for large numbers.
\newcommand{\e}[1]{\cdot 10^{#1}}

% Mathematician's notation for the inner (scalar, dot) product.
\newcommand{\bracket}[1]{\left\langle #1 \right\rangle}
\newcommand{\inner}[2]{\bracket{#1, #2}}

% Placeholders.
\newcommand{\emesswert}{\del{\messwert \pm \messwert}}
\newcommand{\fehlt}{\textcolor{darkred}{Hier fehlen noch Inhalte.}}
\newcommand{\messwert}{\textcolor{blue}{\square}}
\newcommand{\punkte}{\phantom{xxxxx}}
\newcommand{\punktevon}[1]{\begin{flushright}/ #1\end{flushright}}

% Separator for equations on a single line.
\newcommand{\eqnsep}{,\quad}

% Quantum Mechanics
\usepackage{braket}

%%%%%%%%%%%%%%%%%%%%%%%%%%%%%%%%%%%%%%%%%%%%%%%%%%%%%%%%%%%%%%%%%%%%%%%%%%%%%%%
%                                  Headings                                   %
%%%%%%%%%%%%%%%%%%%%%%%%%%%%%%%%%%%%%%%%%%%%%%%%%%%%%%%%%%%%%%%%%%%%%%%%%%%%%%%

% This will set fancy headings to the top of the page. The page number will be
% accompanied by the total number of pages. That way, you will know if any page
% is missing.
%
% If you do not want this for your document, you can just use
% ``\pagestyle{plain}``.

\usepackage{scrpage2}

\pagestyle{scrheadings}
\automark{section}
\cfoot{\footnotesize{Seite \thepage\ / \pageref{LastPage}}}
\chead{}
\ihead{}
\ohead{\rightmark}
\setheadsepline{.4pt}

%%%%%%%%%%%%%%%%%%%%%%%%%%%%%%%%%%%%%%%%%%%%%%%%%%%%%%%%%%%%%%%%%%%%%%%%%%%%%%%
%                            Bibliography (BibTeX)                            %
%%%%%%%%%%%%%%%%%%%%%%%%%%%%%%%%%%%%%%%%%%%%%%%%%%%%%%%%%%%%%%%%%%%%%%%%%%%%%%%

\newcommand{\bibliographyfile}{../../central-bibtex/Central}
\bibliographystyle{apalike2}

%%%%%%%%%%%%%%%%%%%%%%%%%%%%%%%%%%%%%%%%%%%%%%%%%%%%%%%%%%%%%%%%%%%%%%%%%%%%%%%
%                                Abbreviations                                %
%%%%%%%%%%%%%%%%%%%%%%%%%%%%%%%%%%%%%%%%%%%%%%%%%%%%%%%%%%%%%%%%%%%%%%%%%%%%%%%

\newcommand{\dhabk}{\mbox{d.\,h.}}

%%%%%%%%%%%%%%%%%%%%%%%%%%%%%%%%%%%%%%%%%%%%%%%%%%%%%%%%%%%%%%%%%%%%%%%%%%%%%%%
%                                  Licences                                   %
%%%%%%%%%%%%%%%%%%%%%%%%%%%%%%%%%%%%%%%%%%%%%%%%%%%%%%%%%%%%%%%%%%%%%%%%%%%%%%%

\usepackage{ccicons}

\newcommand{\ccbysadetext}{%
	\begin{small}
		Dieses Werk bzw. Inhalt steht unter einer
		\href{http://creativecommons.org/licenses/by-sa/3.0/deed.de}{%
			Creative Commons Namensnennung - Weitergabe unter gleichen
		Bedingungen 3.0 Unported Lizenz}.
	\end{small}
}

\newcommand{\ccbysadetitle}{%
	Lizenz: \href{http://creativecommons.org/licenses/by-sa/3.0/deed.de}
	{CC-BY-SA 3.0 \ccbysa}
}
`` in your main
% document and start with ``\title`` and ``\begin{document}`` afterwards.

% If you need to add additional packages, I recommend not doing this in this
% file, but in your main document. That way, you can just drop in a new
% ``header.tex`` and get all the new commands without having to merge manually.

% Since this file encorporates a CC-BY-SA fragment, this whole files is
% licensed under the CC-BY-SA license.

\documentclass[11pt, ngerman, fleqn, DIV=15, BCOR=2cm, headinclude]{scrartcl}

\usepackage{graphicx}

% Environment to quote the problem. Currently, this is just a new name for the
% quote environment.
\newenvironment{problem}{\begin{quote}\textsf{\textbf{Aufgabenstellung:}}\quad}{\end{quote}}

\setkomafont{caption}{\sf}
\setkomafont{captionlabel}{\usekomafont{caption}}

%%%%%%%%%%%%%%%%%%%%%%%%%%%%%%%%%%%%%%%%%%%%%%%%%%%%%%%%%%%%%%%%%%%%%%%%%%%%%%%
%                                Locale, date                                 %
%%%%%%%%%%%%%%%%%%%%%%%%%%%%%%%%%%%%%%%%%%%%%%%%%%%%%%%%%%%%%%%%%%%%%%%%%%%%%%%

\usepackage{babel}
\usepackage[iso]{isodate}

%%%%%%%%%%%%%%%%%%%%%%%%%%%%%%%%%%%%%%%%%%%%%%%%%%%%%%%%%%%%%%%%%%%%%%%%%%%%%%%
%                          Margins and other spacing                          %
%%%%%%%%%%%%%%%%%%%%%%%%%%%%%%%%%%%%%%%%%%%%%%%%%%%%%%%%%%%%%%%%%%%%%%%%%%%%%%%

\usepackage[parfill]{parskip}
\usepackage{setspace}
\usepackage[activate]{microtype}

\setlength{\columnsep}{2cm}

%%%%%%%%%%%%%%%%%%%%%%%%%%%%%%%%%%%%%%%%%%%%%%%%%%%%%%%%%%%%%%%%%%%%%%%%%%%%%%%
%                                    Color                                    %
%%%%%%%%%%%%%%%%%%%%%%%%%%%%%%%%%%%%%%%%%%%%%%%%%%%%%%%%%%%%%%%%%%%%%%%%%%%%%%%

\usepackage[usenames, dvipsnames]{xcolor}

\colorlet{darkred}{red!70!black}
\colorlet{darkblue}{blue!70!black}
\colorlet{darkgreen}{green!40!black}

%%%%%%%%%%%%%%%%%%%%%%%%%%%%%%%%%%%%%%%%%%%%%%%%%%%%%%%%%%%%%%%%%%%%%%%%%%%%%%%
%                         Font and font like settings                         %
%%%%%%%%%%%%%%%%%%%%%%%%%%%%%%%%%%%%%%%%%%%%%%%%%%%%%%%%%%%%%%%%%%%%%%%%%%%%%%%

% This replaces all fonts with Bitstream Charter, Bitstream Vera Sans and
% Bitstream Vera Mono. Math will be rendered in Charter.
\usepackage[charter, greekuppercase=italicized]{mathdesign}
\usepackage{beramono}
\usepackage{berasans}

% Bold, sans-serif tensors. This fragment is taken from “egreg” from
% http://tex.stackexchange.com/a/82747/8945 and licensed under `CC-BY-SA
% <https://creativecommons.org/licenses/by-sa/3.0/>`_.
\usepackage{bm}
\DeclareMathAlphabet{\mathsfit}{\encodingdefault}{\sfdefault}{m}{sl}
\SetMathAlphabet{\mathsfit}{bold}{\encodingdefault}{\sfdefault}{bx}{sl}
\newcommand{\tens}[1]{\bm{\mathsfit{#1}}}

% Bold vectors.
\renewcommand{\vec}[1]{\boldsymbol{#1}}

%%%%%%%%%%%%%%%%%%%%%%%%%%%%%%%%%%%%%%%%%%%%%%%%%%%%%%%%%%%%%%%%%%%%%%%%%%%%%%%
%                               Input encoding                                %
%%%%%%%%%%%%%%%%%%%%%%%%%%%%%%%%%%%%%%%%%%%%%%%%%%%%%%%%%%%%%%%%%%%%%%%%%%%%%%%

\usepackage[T1]{fontenc}
\usepackage[utf8]{inputenc}

%%%%%%%%%%%%%%%%%%%%%%%%%%%%%%%%%%%%%%%%%%%%%%%%%%%%%%%%%%%%%%%%%%%%%%%%%%%%%%%
%                         Hyperrefs and PDF metadata                          %
%%%%%%%%%%%%%%%%%%%%%%%%%%%%%%%%%%%%%%%%%%%%%%%%%%%%%%%%%%%%%%%%%%%%%%%%%%%%%%%

\usepackage{hyperref}
\usepackage{lastpage}

% This sets the author in the properties of the PDF as well. If you want to
% change it, just override it with another ``\hypersetup`` call.
\hypersetup{
	breaklinks=false,
	citecolor=darkgreen,
	colorlinks=true,
	linkcolor=darkblue,
	menucolor=black,
	pdfauthor={Martin Ueding},
	urlcolor=darkblue,
}

%%%%%%%%%%%%%%%%%%%%%%%%%%%%%%%%%%%%%%%%%%%%%%%%%%%%%%%%%%%%%%%%%%%%%%%%%%%%%%%
%                               Math Operators                                %
%%%%%%%%%%%%%%%%%%%%%%%%%%%%%%%%%%%%%%%%%%%%%%%%%%%%%%%%%%%%%%%%%%%%%%%%%%%%%%%

% AMS environments like ``align`` and theorems like ``proof``.
\usepackage{amsmath}
\usepackage{amsthm}

% Common math constructs like partial derivatives.
\usepackage{commath}

% Physical units.
\usepackage[output-decimal-marker={,}]{siunitx}

% Since I use mathdesign with italic uppercase greek characters, the Ohm unit will be displayed with an italic Ω by default. Units have to be roman, so this forces it the right way.
\DeclareSIUnit{\ohm}{$\Omegaup$}
\DeclareSIUnit{\division}{DIV}
\DeclareSIUnit{\voltss}{$\mathrm{V_{SS}}$}

% Word like operators.
\DeclareMathOperator{\acosh}{arcosh}
\DeclareMathOperator{\arcosh}{arcosh}
\DeclareMathOperator{\arcsinh}{arsinh}
\DeclareMathOperator{\arsinh}{arsinh}
\DeclareMathOperator{\asinh}{arsinh}
\DeclareMathOperator{\card}{card}
\DeclareMathOperator{\csch}{cshs}
\DeclareMathOperator{\diam}{diam}
\DeclareMathOperator{\sech}{sech}
\renewcommand{\Im}{\mathop{{}\mathrm{Im}}\nolimits}
\renewcommand{\Re}{\mathop{{}\mathrm{Re}}\nolimits}

% Fourier transform.
\DeclareMathOperator{\fourier}{\ensuremath{\mathcal{F}}}

% Roman versions of “e” and “i” to serve as Euler's number and the imaginary
% constant.
\newcommand{\ee}{\eup}
\newcommand{\eup}{\mathrm e}
\newcommand{\ii}{\iup}
\newcommand{\iup}{\mathrm i}

% Symbols for the various mathematical fields (natural numbers, integers,
% rational numbers, real numbers, complex numbers).
\newcommand{\C}{\ensuremath{\mathbb C}}
\newcommand{\N}{\ensuremath{\mathbb N}}
\newcommand{\Q}{\ensuremath{\mathbb Q}}
\newcommand{\R}{\ensuremath{\mathbb R}}
\newcommand{\Z}{\ensuremath{\mathbb Z}}

% Shape like operators.
\DeclareMathOperator{\dalambert}{\Box}
\DeclareMathOperator{\laplace}{\bigtriangleup}
\newcommand{\curl}{\vnabla \times}
\newcommand{\divergence}[1]{\inner{\vnabla}{#1}}
\newcommand{\vnabla}{\vec \nabla}

\newcommand{\half}{\frac 12}

% Unit vector (German „Einheitsvektor“).
\newcommand{\ev}{\hat{\vec e}}

% Scientific notation for large numbers.
\newcommand{\e}[1]{\cdot 10^{#1}}

% Mathematician's notation for the inner (scalar, dot) product.
\newcommand{\bracket}[1]{\left\langle #1 \right\rangle}
\newcommand{\inner}[2]{\bracket{#1, #2}}

% Placeholders.
\newcommand{\emesswert}{\del{\messwert \pm \messwert}}
\newcommand{\fehlt}{\textcolor{darkred}{Hier fehlen noch Inhalte.}}
\newcommand{\messwert}{\textcolor{blue}{\square}}
\newcommand{\punkte}{\phantom{xxxxx}}
\newcommand{\punktevon}[1]{\begin{flushright}/ #1\end{flushright}}

% Separator for equations on a single line.
\newcommand{\eqnsep}{,\quad}

% Quantum Mechanics
\usepackage{braket}

%%%%%%%%%%%%%%%%%%%%%%%%%%%%%%%%%%%%%%%%%%%%%%%%%%%%%%%%%%%%%%%%%%%%%%%%%%%%%%%
%                                  Headings                                   %
%%%%%%%%%%%%%%%%%%%%%%%%%%%%%%%%%%%%%%%%%%%%%%%%%%%%%%%%%%%%%%%%%%%%%%%%%%%%%%%

% This will set fancy headings to the top of the page. The page number will be
% accompanied by the total number of pages. That way, you will know if any page
% is missing.
%
% If you do not want this for your document, you can just use
% ``\pagestyle{plain}``.

\usepackage{scrpage2}

\pagestyle{scrheadings}
\automark{section}
\cfoot{\footnotesize{Seite \thepage\ / \pageref{LastPage}}}
\chead{}
\ihead{}
\ohead{\rightmark}
\setheadsepline{.4pt}

%%%%%%%%%%%%%%%%%%%%%%%%%%%%%%%%%%%%%%%%%%%%%%%%%%%%%%%%%%%%%%%%%%%%%%%%%%%%%%%
%                            Bibliography (BibTeX)                            %
%%%%%%%%%%%%%%%%%%%%%%%%%%%%%%%%%%%%%%%%%%%%%%%%%%%%%%%%%%%%%%%%%%%%%%%%%%%%%%%

\newcommand{\bibliographyfile}{../../central-bibtex/Central}
\bibliographystyle{apalike2}

%%%%%%%%%%%%%%%%%%%%%%%%%%%%%%%%%%%%%%%%%%%%%%%%%%%%%%%%%%%%%%%%%%%%%%%%%%%%%%%
%                                Abbreviations                                %
%%%%%%%%%%%%%%%%%%%%%%%%%%%%%%%%%%%%%%%%%%%%%%%%%%%%%%%%%%%%%%%%%%%%%%%%%%%%%%%

\newcommand{\dhabk}{\mbox{d.\,h.}}

%%%%%%%%%%%%%%%%%%%%%%%%%%%%%%%%%%%%%%%%%%%%%%%%%%%%%%%%%%%%%%%%%%%%%%%%%%%%%%%
%                                  Licences                                   %
%%%%%%%%%%%%%%%%%%%%%%%%%%%%%%%%%%%%%%%%%%%%%%%%%%%%%%%%%%%%%%%%%%%%%%%%%%%%%%%

\usepackage{ccicons}

\newcommand{\ccbysadetext}{%
	\begin{small}
		Dieses Werk bzw. Inhalt steht unter einer
		\href{http://creativecommons.org/licenses/by-sa/3.0/deed.de}{%
			Creative Commons Namensnennung - Weitergabe unter gleichen
		Bedingungen 3.0 Unported Lizenz}.
	\end{small}
}

\newcommand{\ccbysadetitle}{%
	Lizenz: \href{http://creativecommons.org/licenses/by-sa/3.0/deed.de}
	{CC-BY-SA 3.0 \ccbysa}
}
`` in your main
% document and start with ``\title`` and ``\begin{document}`` afterwards.

% If you need to add additional packages, I recommend not doing this in this
% file, but in your main document. That way, you can just drop in a new
% ``header.tex`` and get all the new commands without having to merge manually.

% Since this file encorporates a CC-BY-SA fragment, this whole files is
% licensed under the CC-BY-SA license.

\documentclass[11pt, ngerman, fleqn, DIV=15, BCOR=2cm, headinclude]{scrartcl}

\usepackage{graphicx}

% Environment to quote the problem. Currently, this is just a new name for the
% quote environment.
\newenvironment{problem}{\begin{quote}\textsf{\textbf{Aufgabenstellung:}}\quad}{\end{quote}}

\setkomafont{caption}{\sf}
\setkomafont{captionlabel}{\usekomafont{caption}}

%%%%%%%%%%%%%%%%%%%%%%%%%%%%%%%%%%%%%%%%%%%%%%%%%%%%%%%%%%%%%%%%%%%%%%%%%%%%%%%
%                                Locale, date                                 %
%%%%%%%%%%%%%%%%%%%%%%%%%%%%%%%%%%%%%%%%%%%%%%%%%%%%%%%%%%%%%%%%%%%%%%%%%%%%%%%

\usepackage{babel}
\usepackage[iso]{isodate}

%%%%%%%%%%%%%%%%%%%%%%%%%%%%%%%%%%%%%%%%%%%%%%%%%%%%%%%%%%%%%%%%%%%%%%%%%%%%%%%
%                          Margins and other spacing                          %
%%%%%%%%%%%%%%%%%%%%%%%%%%%%%%%%%%%%%%%%%%%%%%%%%%%%%%%%%%%%%%%%%%%%%%%%%%%%%%%

\usepackage[parfill]{parskip}
\usepackage{setspace}
\usepackage[activate]{microtype}

\setlength{\columnsep}{2cm}

%%%%%%%%%%%%%%%%%%%%%%%%%%%%%%%%%%%%%%%%%%%%%%%%%%%%%%%%%%%%%%%%%%%%%%%%%%%%%%%
%                                    Color                                    %
%%%%%%%%%%%%%%%%%%%%%%%%%%%%%%%%%%%%%%%%%%%%%%%%%%%%%%%%%%%%%%%%%%%%%%%%%%%%%%%

\usepackage[usenames, dvipsnames]{xcolor}

\colorlet{darkred}{red!70!black}
\colorlet{darkblue}{blue!70!black}
\colorlet{darkgreen}{green!40!black}

%%%%%%%%%%%%%%%%%%%%%%%%%%%%%%%%%%%%%%%%%%%%%%%%%%%%%%%%%%%%%%%%%%%%%%%%%%%%%%%
%                         Font and font like settings                         %
%%%%%%%%%%%%%%%%%%%%%%%%%%%%%%%%%%%%%%%%%%%%%%%%%%%%%%%%%%%%%%%%%%%%%%%%%%%%%%%

% This replaces all fonts with Bitstream Charter, Bitstream Vera Sans and
% Bitstream Vera Mono. Math will be rendered in Charter.
\usepackage[charter, greekuppercase=italicized]{mathdesign}
\usepackage{beramono}
\usepackage{berasans}

% Bold, sans-serif tensors. This fragment is taken from “egreg” from
% http://tex.stackexchange.com/a/82747/8945 and licensed under `CC-BY-SA
% <https://creativecommons.org/licenses/by-sa/3.0/>`_.
\usepackage{bm}
\DeclareMathAlphabet{\mathsfit}{\encodingdefault}{\sfdefault}{m}{sl}
\SetMathAlphabet{\mathsfit}{bold}{\encodingdefault}{\sfdefault}{bx}{sl}
\newcommand{\tens}[1]{\bm{\mathsfit{#1}}}

% Bold vectors.
\renewcommand{\vec}[1]{\boldsymbol{#1}}

%%%%%%%%%%%%%%%%%%%%%%%%%%%%%%%%%%%%%%%%%%%%%%%%%%%%%%%%%%%%%%%%%%%%%%%%%%%%%%%
%                               Input encoding                                %
%%%%%%%%%%%%%%%%%%%%%%%%%%%%%%%%%%%%%%%%%%%%%%%%%%%%%%%%%%%%%%%%%%%%%%%%%%%%%%%

\usepackage[T1]{fontenc}
\usepackage[utf8]{inputenc}

%%%%%%%%%%%%%%%%%%%%%%%%%%%%%%%%%%%%%%%%%%%%%%%%%%%%%%%%%%%%%%%%%%%%%%%%%%%%%%%
%                         Hyperrefs and PDF metadata                          %
%%%%%%%%%%%%%%%%%%%%%%%%%%%%%%%%%%%%%%%%%%%%%%%%%%%%%%%%%%%%%%%%%%%%%%%%%%%%%%%

\usepackage{hyperref}
\usepackage{lastpage}

% This sets the author in the properties of the PDF as well. If you want to
% change it, just override it with another ``\hypersetup`` call.
\hypersetup{
	breaklinks=false,
	citecolor=darkgreen,
	colorlinks=true,
	linkcolor=darkblue,
	menucolor=black,
	pdfauthor={Martin Ueding},
	urlcolor=darkblue,
}

%%%%%%%%%%%%%%%%%%%%%%%%%%%%%%%%%%%%%%%%%%%%%%%%%%%%%%%%%%%%%%%%%%%%%%%%%%%%%%%
%                               Math Operators                                %
%%%%%%%%%%%%%%%%%%%%%%%%%%%%%%%%%%%%%%%%%%%%%%%%%%%%%%%%%%%%%%%%%%%%%%%%%%%%%%%

% AMS environments like ``align`` and theorems like ``proof``.
\usepackage{amsmath}
\usepackage{amsthm}

% Common math constructs like partial derivatives.
\usepackage{commath}

% Physical units.
\usepackage[output-decimal-marker={,}]{siunitx}

% Since I use mathdesign with italic uppercase greek characters, the Ohm unit will be displayed with an italic Ω by default. Units have to be roman, so this forces it the right way.
\DeclareSIUnit{\ohm}{$\Omegaup$}
\DeclareSIUnit{\division}{DIV}
\DeclareSIUnit{\voltss}{$\mathrm{V_{SS}}$}

% Word like operators.
\DeclareMathOperator{\acosh}{arcosh}
\DeclareMathOperator{\arcosh}{arcosh}
\DeclareMathOperator{\arcsinh}{arsinh}
\DeclareMathOperator{\arsinh}{arsinh}
\DeclareMathOperator{\asinh}{arsinh}
\DeclareMathOperator{\card}{card}
\DeclareMathOperator{\csch}{cshs}
\DeclareMathOperator{\diam}{diam}
\DeclareMathOperator{\sech}{sech}
\renewcommand{\Im}{\mathop{{}\mathrm{Im}}\nolimits}
\renewcommand{\Re}{\mathop{{}\mathrm{Re}}\nolimits}

% Fourier transform.
\DeclareMathOperator{\fourier}{\ensuremath{\mathcal{F}}}

% Roman versions of “e” and “i” to serve as Euler's number and the imaginary
% constant.
\newcommand{\ee}{\eup}
\newcommand{\eup}{\mathrm e}
\newcommand{\ii}{\iup}
\newcommand{\iup}{\mathrm i}

% Symbols for the various mathematical fields (natural numbers, integers,
% rational numbers, real numbers, complex numbers).
\newcommand{\C}{\ensuremath{\mathbb C}}
\newcommand{\N}{\ensuremath{\mathbb N}}
\newcommand{\Q}{\ensuremath{\mathbb Q}}
\newcommand{\R}{\ensuremath{\mathbb R}}
\newcommand{\Z}{\ensuremath{\mathbb Z}}

% Shape like operators.
\DeclareMathOperator{\dalambert}{\Box}
\DeclareMathOperator{\laplace}{\bigtriangleup}
\newcommand{\curl}{\vnabla \times}
\newcommand{\divergence}[1]{\inner{\vnabla}{#1}}
\newcommand{\vnabla}{\vec \nabla}

\newcommand{\half}{\frac 12}

% Unit vector (German „Einheitsvektor“).
\newcommand{\ev}{\hat{\vec e}}

% Scientific notation for large numbers.
\newcommand{\e}[1]{\cdot 10^{#1}}

% Mathematician's notation for the inner (scalar, dot) product.
\newcommand{\bracket}[1]{\left\langle #1 \right\rangle}
\newcommand{\inner}[2]{\bracket{#1, #2}}

% Placeholders.
\newcommand{\emesswert}{\del{\messwert \pm \messwert}}
\newcommand{\fehlt}{\textcolor{darkred}{Hier fehlen noch Inhalte.}}
\newcommand{\messwert}{\textcolor{blue}{\square}}
\newcommand{\punkte}{\phantom{xxxxx}}
\newcommand{\punktevon}[1]{\begin{flushright}/ #1\end{flushright}}

% Separator for equations on a single line.
\newcommand{\eqnsep}{,\quad}

% Quantum Mechanics
\usepackage{braket}

%%%%%%%%%%%%%%%%%%%%%%%%%%%%%%%%%%%%%%%%%%%%%%%%%%%%%%%%%%%%%%%%%%%%%%%%%%%%%%%
%                                  Headings                                   %
%%%%%%%%%%%%%%%%%%%%%%%%%%%%%%%%%%%%%%%%%%%%%%%%%%%%%%%%%%%%%%%%%%%%%%%%%%%%%%%

% This will set fancy headings to the top of the page. The page number will be
% accompanied by the total number of pages. That way, you will know if any page
% is missing.
%
% If you do not want this for your document, you can just use
% ``\pagestyle{plain}``.

\usepackage{scrpage2}

\pagestyle{scrheadings}
\automark{section}
\cfoot{\footnotesize{Seite \thepage\ / \pageref{LastPage}}}
\chead{}
\ihead{}
\ohead{\rightmark}
\setheadsepline{.4pt}

%%%%%%%%%%%%%%%%%%%%%%%%%%%%%%%%%%%%%%%%%%%%%%%%%%%%%%%%%%%%%%%%%%%%%%%%%%%%%%%
%                            Bibliography (BibTeX)                            %
%%%%%%%%%%%%%%%%%%%%%%%%%%%%%%%%%%%%%%%%%%%%%%%%%%%%%%%%%%%%%%%%%%%%%%%%%%%%%%%

\newcommand{\bibliographyfile}{../../central-bibtex/Central}
\bibliographystyle{apalike2}

%%%%%%%%%%%%%%%%%%%%%%%%%%%%%%%%%%%%%%%%%%%%%%%%%%%%%%%%%%%%%%%%%%%%%%%%%%%%%%%
%                                Abbreviations                                %
%%%%%%%%%%%%%%%%%%%%%%%%%%%%%%%%%%%%%%%%%%%%%%%%%%%%%%%%%%%%%%%%%%%%%%%%%%%%%%%

\newcommand{\dhabk}{\mbox{d.\,h.}}

%%%%%%%%%%%%%%%%%%%%%%%%%%%%%%%%%%%%%%%%%%%%%%%%%%%%%%%%%%%%%%%%%%%%%%%%%%%%%%%
%                                  Licences                                   %
%%%%%%%%%%%%%%%%%%%%%%%%%%%%%%%%%%%%%%%%%%%%%%%%%%%%%%%%%%%%%%%%%%%%%%%%%%%%%%%

\usepackage{ccicons}

\newcommand{\ccbysadetext}{%
	\begin{small}
		Dieses Werk bzw. Inhalt steht unter einer
		\href{http://creativecommons.org/licenses/by-sa/3.0/deed.de}{%
			Creative Commons Namensnennung - Weitergabe unter gleichen
		Bedingungen 3.0 Unported Lizenz}.
	\end{small}
}

\newcommand{\ccbysadetitle}{%
	Lizenz: \href{http://creativecommons.org/licenses/by-sa/3.0/deed.de}
	{CC-BY-SA 3.0 \ccbysa}
}


\usepackage{booktabs}
\usepackage{minted}
\usepackage{placeins}
\usepackage{xfrac}

\newcommand\versuchsnummer{443}

\ihead{physik412 – Versuch \versuchsnummer}
\ifoot{Martin Ueding, Lino Lemmer}

\subject{Praktikumsprotokoll}
\title{Kernmagnetische Relaxation}
\subtitle{physik412 – Versuch \versuchsnummer}
\author{
    Martin Ueding \\
    \small{\href{mailto:mu@martin-ueding.de}{mu@martin-ueding.de}}
    \and
    Lino Lemmer \\
    \small{\href{mailto:s6lilemm@uni-bonn.de}{s6lilemm@uni-bonn.de}}
}

\setcounter{secnumdepth}{4}
\setcounter{tocdepth}{4}

\begin{document}

\maketitle

\begin{abstract}
    \fehlt
\end{abstract}

\tableofcontents

%%%%%%%%%%%%%%%%%%%%%%%%%%%%%%%%%%%%%%%%%%%%%%%%%%%%%%%%%%%%%%%%%%%%%%%%%%%%%%%
%                                   Theorie                                   %
%%%%%%%%%%%%%%%%%%%%%%%%%%%%%%%%%%%%%%%%%%%%%%%%%%%%%%%%%%%%%%%%%%%%%%%%%%%%%%%

\chapter{Theorie}

\cite[Abschnitt~15.9.3 „Magnetische~Resonanz”]{meschede-gerthsen_24}

\cite{physik412-Anleitung}

\section{Spin $\sfrac12$ Teilchen im homogenen Magnetfeld}

In diesem Versuch untersuchen wie die magnetische Resonanz von Protonen, die in
Form von Wasserstoff in einer großen Form in der Probe aus leichtem Mineralöl
vorliegen. Die Protonen zeichnen sich durch einen Spin $\sfrac12$ aus.

Dieser Spin wurde in „Experimentalphysik 4“ eingeführt und verhält sich wie ein
Drehimpuls. Die Quantenzahl $s$ ist hier immer $\sfrac12$. Es bleibt die
$z$-Komponente $m_s$, die die Werte $-\sfrac12$ und $\sfrac12$ annehmen kann.

Die Protonen besitzen ein magnetisches Moment $\vec\mu$, das immer parallel zum
Spin ist und eine feste Magnitude besitzt:
\[
    \mu = \gamma \hbar s
\]

Der Faktor $\gamma$ ist das gyromagnetische Verhältnis und ist für verschiedene
Teilchen (z.\,B. Protonen, Elektronen) unterschiedlich.

\subsection{Aufspaltung in zwei Zustände}

In einem homogenen Magnetfeld $\vec B_0$ erfährt das magnetische Moment
$\vec\mu$ ein Drehmoment $\vec M = \vec \mu \times \vec B_0$, das zu einer
Einstellenergie führt:
\[
    E = - \inner{\vec \mu}{\vec B_0}
\]

Da die magnetische Spinquantenzahl $m_s$ nur die Werte $-\sfrac12$ und
$\sfrac12$ annehmen darf, gibt es nur zwei Zustände, $\ket\uparrow$ und
$\ket\downarrow$, die aufgrund der Einstellenergie unterschiedliche Energie
haben. Ohne das externe Magnetfeld wären diese Zustände entartet.

\subsection{Besetzung, thermisches Gleichgewicht}

Ohne das externe Feld liegen alle Spins in einem beliebigen Zustand vor, so
dass es keine makroskopische Magnetisierung $\vec M$ gibt.

\subsection{Longitudinale Relaxation (Relaxationszeit $T_1$)}

Ist das System in einem beliebigen Zustand, so werden die Spins nach einiger
Zeit in den Grundzustand wechseln. Die Rate, mit der die Spins wechseln ist
proportional zur Anzahl, die noch nicht gewechselt sind. Aus diesem Ansatz
lässt sich mit geeigneten Anfangsbedingungen folgende Zeitentwicklung
herleiten: \parencite[Formel~P443.2]{physik412-Anleitung}
\begin{equation}
    \label{eq:M/exp}
    \vec M(t) = \vec M_0 \del{1 - \exp\del{- \frac{t}{T_1}}}
\end{equation}

Dabei haben wir die Zeitkonstante $T_1$ eingeführt, die beschreibt, wie schnell
diese Relaxation vor sich geht. Diese Zeit ist relativ lang.

\subsection{Präzession um $z$–Achse}
\label{subsec:Präzession}

Da das magnetische Moment $\vec\mu$ wahrscheinlich nicht parallel zum
magnetischen Feld ist, wird auf es konstant das Drehmoment durch das
magnetische Feld wirken. Dies führt dazu, dass das magnetische Moment um die
$z$-Achse präzessiert. Die Frequenz dieser Präzession ist die Larmorfrequenz:
\[
    \omega_\text L = \gamma B_0
\]

Die makroskopische Magnetisierung, die sich als Mittel aus allen magnetischen
Momenten ergibt, wird daher auch um die $z$-Achse präzessieren, zumindest
solange die einzelnen Spins mit der gleichen Frequenz präzessieren. Dazu später
mehr.

\section{Reaktion von Spin auf zusätzliches Feld}

In diesem Experiment werden wir zusätzlich zum (möglichst) homogenen Feld $\vec
B_0$ ein gepulstes, sich drehendes, Feld $B_\text{RF}$ verwenden, dessen
Frequenz im Radiobereich (\si{\mega\hertz}) liegt.

\subsection{Blochkugel}

Die Spins haben im externen Feld die Basiszustände $\ket\uparrow$ und
$\ket\downarrow$. Es können aber auch Zustände auftreten, die sich durch zwei
Mischwinkel $\theta$ und $\phi$ charakterisieren lassen:
\parencite{wikipedia/bloch_kugel}
\[
    \ket{\theta, \phi}
    = \cos\del{\frac\theta2} \ket\uparrow
    + \eup^{\iup \phi} \sin\del{\frac\theta2} \ket\downarrow
\]

Durch diese Superposition ist es möglich, dass das magnetische Moment in jede
beliebige Richtung liegen kann. Die energetisch günstigste Variante ist jedoch,
wenn der Spin möglichst parallel zum Magnetfeld $\vec B_0$ liegt, also
$\ket\uparrow$.

Im weiteren Verlauf können wir uns die Orientierung des Spins als
Einheitsvektor in der Kugel vorstellen.

\subsection{Rotierendes Bezugssystem}

Wie schon in \ref{subsec:Präzession} beschrieben, präzessiert das magnetische
Moment und der Spin um die $z$-Achse um ruhenden Inertialsystem des Labors
$\Sigma$. Es ist für die weiteren Überlegungen hilfreich, ein rotierendes
Bezugssystem $\Sigma^*$ zu betrachten, in dem das magnetische Moment fest ist.

In einem solchen rotierenden Bezugssystem tritt noch eine weitere,
transversales magnetische Flussdichte auf. Diese hat genau die Stärke, dass
sich die resultierende Flussdichte genau in Richtung des magnetischen Momentes
befindet. Diese präzessiert dann um sich selbst, es bleibt also stehen. Das
Bezugssystem war gerade so konstruiert. \parencite{teach_spin_manual}

Mit einer weiteren Spule können wir weitere transversale Magnetfelder erzeugen.
Durch einen kurzen Puls kann die im System $\Sigma^*$ beobachtete magnetische
Flussdichte so verändert werden, dass die parallel zur $x^*$-Achse liegt.

\subsection{$\piup/2$–Puls}

Ein Puls des rotierenden Magnetfeldes der richtigen Stärke und Länge wird dazu
führen, dass das magnetische Moment für den Augenblick nur um die mitgeführte
$x$-Achse, also die $x^*$-Achse präzessiert und somit von der $z^*$ auf die
$y^*$-Achse kippt. Ist dies erreicht, muss der Puls beendet sein, damit das
Moment nicht noch weiter kippt.

Das magnetische Moment einer überwiegenden Anzahl Teilchen liegt nun auf der
$x^*$-Achse. Somit liegt auch die Magnetisierung $\vec M$ auf der $x^*$-Achse.
Da sich das Bezugssystem um die $z$-Achse dreht, ist auf der (festen) $x$-Achse
nun eine makroskopische Magnetisierung zu beobachten, die sich zeitlich
sinusförmig verhält.

Mit der Messspule ist nun die mit der Larmorfrequenz oszillierende
Magnetisierung zu beobachten.

An dieser Stelle ist \cite[Abbildung~15.55 auf Seite~765]{meschede-gerthsen_24}
recht illustrativ.

\subsection{$\piup$–Puls}

Ähnlich wie der $\piup/2$-Puls kippt der $\piup$-Puls das magnetische Moment,
jedoch um den doppelten Winkel. Im thermalisierten Grundzustand des Systems
invertiert dieser Puls $\ket\uparrow$ und $\ket\downarrow$. Geht dem
$\piup$-Puls ein $\piup/2$-Puls voraus, so wird der Azimutwinkel $\phi$ des
Zustandes verändert.

\subsection{Transversale Relaxation}

Die Larmorfrequenz an jeder Stelle fast gleich, jedoch nur fast. Dadurch
präzessieren die einzelnen Spins mit einer lokal leicht verschiedenen
Geschwindigkeit. Auch wenn die Unterschiede nur klein sind, reicht es aus, wenn
die relative Verschiebung zwischen den Spins $\piup$ beträgt, dass keine
makroskopische transversale Magnetisierung gemessen werden kann.

Zum einen wird die Ortsabhängigkeit der Larmorfrequenz durch das leicht
inhomogene Magnetfeld des großen äußeren Magneten verursacht. Diese Veränderung
ist konstant und kann, wie wir später beschreiben werden, mit der
Hahn-Spinecho-Sequenz rückgängig gemacht werden.

Außerdem erzeugen die magnetischen Momente der einzelnen Spins für sich auch
wieder Magnetfelder $\vec H$, die auch zu einer Verschiebung der Larmorfrequenz
führen. Dieser Effekt ist stark zeitabhängig und kann nicht korrigiert werden.

Da es diese verschiedenen Effekte gibt, werden zwei verschiedene
Relaxationszeiten eingeführt.

\subsubsection{Homogene Transversale Relaxationszeit $T_2$}

Durch einen $\piup/2$-Puls wird die makroskopische Magnetisierung in die
$x$-$y$-Ebene gekippt und präzessiert dann mit der Larmorfrequenz um die
$z$-Achse, jedoch zerläuft diese Magnetisierung aufgrund der leicht
verschiedenen Larmorfrequenzen. Dies nennt man „freien Induktionszerfall“, kurz
„FID“\footnote{„FID“ kommt von „free induction decay”.}.

Wäre das Magnetfeld perfekt homogen, so basierte die Frequenzverschiebung nur
auf dem Magnetfeld der anderen Spins. Diese homogene Relaxation ist die
eigentlich interessante Zeitkonstante, da sie vom nur Material abhängt und
nicht vom experimentellen Aufbau.

Mit der Hahn-Spinecho-Sequenz können wir später diese Zeit unabhängig vom
externen Magnetfeld messen.

Hier ist ein exponentieller Abfall der Gesamtmagnetisierung zu erwarten, die
wir mit der Zeitkonstante $T_2$ beschreiben werden.

\subsubsection{Effektive Transversale Relaxationszeit $T_2^*$}

Das inhomogene Magnetfeld verstärkt den freien Induktionszerfall. Auch hier ist
ein exponentieller Abfall zu beobachten, der jedoch schneller ist, als der
Effekt durch die homogene Relaxation alleine. Daher ist $T_2^*$ kürzer als
$T_2$.

Mit der inhomogenen transversalen Relaxationszeit $T_{2,\text{inhom}}$ hängen
die anderen wie folgt zusammen: \parencite[Formel~P443.5]{physik412-Anleitung}
\begin{equation}
    \label{eq:}
    \frac{1}{T_2^*} = \frac 1{T_2} + \frac{1}{T_{2,\text{inhom}}}
\end{equation}

\subsection{Rabi-Oszillation}

Rabi-Oszillationen sind allgemein Oszillationen eines Zweizustandsystems durch
eine externe Anregung. \cite{wikipedia/Rabi_Oszillation}

Hier ist diese externe Anregung der RF-Puls, den wir durch die Spule schicken.
Dabei legen wir die Frequenz des Pulses möglichst genau auf die Larmorfrequenz,
und variieren hier die Pulslänge. Wir erwarten, dass sich bei einem sehr kurzen
Puls die makroskopische Magnetisierung nicht verändert. Wenn der Puls länger
wird, und in den Bereich des $\piup/2$-Puls kommt, ist ein maximales Signal zu
erwarten. Wird der Puls länger, so dass er in der Nähe des $\piup$-Pulses
kommt, ist wieder kein Signal zu erwarten. In der Größenordnung eines
$3\piup/4$-Pulses sollte die Amplitude wieder maximal sein, jedoch die Phase
umgekehrt sein.

\cite[Abschnitt~15.9.5 „Rabi-Atomstrahlresonanz”]{meschede-gerthsen_24}

\section{Messmethoden}

In diesem Abschnitt möchten wir die Grundlage der Messmethoden erläutern, die
wir nachher bei der Durchführung gebrauchen werden.

\subsection{Konstante Grundspannung}

Wir betreiben das Oszilloskop mit DC-Kopplung, so dass allen Messung ein
konstante Grundspannung zugrundeliegen kann, die wir bei jeder Messung noch
abziehen müssen.

\subsection{Longitudinale Relaxationszeit $T_1$}

\subsubsection{Sättigungs–Zurückgewinnung}

Hier geben wir einen $\piup/2$ Puls und nach einer Verzögerungszeit $\tau$ noch
einen weiteren $\piup/2$-Puls. Die Magnetisierung kippt durch den ersten Puls in
die transversale Ebene. Dort fällt die Magnetisierung gemäß der longitudinalen
Relaxation wieder in den Grundzustand zurück. Bevor dies zu stark passiert,
kippen wir um weitere $\piup/2$ und messen dann das Signal.

Für sehr kleine Verzögerungszeiten $\tau$ erwarten wir kein Signal, da die
Magnetisierung dann um $\piup$ gekippt immer noch parallel zur $z$-Achse ist.
Somit verschwindet der transversale Anteil. Je länger die Verzögerungszeit ist,
desto mehr kann die Magnetisierung wieder zur $z$-Achse zurückkehren und nach
dem letzten Puls als transversaler Anteil gemessen werden.

Da während der Verzögerungszeit eine longitudinale Relaxation gemäß
\eqref{eq:M/exp} stattfindet, erwarten wir genau diese Zeitentwicklung bei
dieser Methode.

Durch Anpassen der Funktion an unsere Messwerte erhalten wir $T_1$.

\subsubsection{Polarisations–Zurückgewinnung}

Hier wird zuerst ein $\piup$-Puls geschickt, nach der Verzögerungszeit ein
$\piup/2$-Puls. Für sehr kurze Zeiten $\tau$ entspricht dies einem
$3\piup/4$-Puls, so dass wir hier die volle Magnetisierung messen, jedoch mit
entgegengesetzter Phase.

Wenn die Verzögerungszeit $\tau$ gerade so bemessen ist, dass die Magnetisierung von
antiparallel zu transversal relaxiert ist, wird der letzte Puls diese wieder
antiparallel stellen, wir werden kein Signal messen.

Für größere Verzögerungszeiten konnte die Magnetisierung wieder in Richtung der
$z$-Achse wandern, so dass wir dann ein Signal messen können.

Aufgrund dessen erwarten wir zwar auch eine exponenzielle Annäherung an den
Maximalwert, jedoch beginnend beim negativen Maximalwert:
\parencite[Formel~P443.4]{physik412-Anleitung}
\begin{equation}
    \label{eq:M/exp/2}
    \vec M(t) = \vec M_0 \del{1 - 2 \exp\del{- \frac{t}{T_1}}}
\end{equation}

\subsection{Transversale Relaxationszeit $T_2$}

Bei der transversalen Relaxationszeit müssen die Effektive und die Homogene
unterschieden werden. Zuerst werden wir Effektive behandeln.

\subsubsection{FID-Signal}

Betrachten wir ein einzelnes FID-Signal nach einem $\piup/2$-Puls, so fällt dies
näherungsweise exponenziell ab. An diesen Abfall können wir eine
Exponentialfunktion anpassen. Aus der Anpassung erhalten wir dann die effektive
transversale Relaxationszeit.

\subsubsection{Hahn–Spinecho–Sequenz}

Um die von der Zeit unabhängigen Inhomogenitäten eliminieren zu können, benutzt
man das Hahn'sche Spinecho. Dazu wird mit einem $\piup/2$-Puls die
Magnetisierung in die $x$-$y$-Ebene gedreht. Die Spins präzessieren nun mit
unterschiedlichen Larmorfrequenzen.

Nach einer Verzögerungszeit $\tau$ wird ein $\piup$-Puls geschickt, der alle
Spins in der transversalen Ebene dreht. Die Windung, die die Spins bis zu
dieser Stelle „aufgedreht“ haben, drehen sie jetzt wieder zurück. Auf diese
Weise werden sie sich nach einer weiteren Periode der Länge $\tau$ alle in
einem Winkel treffen, da sie mit der gleichen, lokal jedoch abweichenden,
Larmorfrequenz präzessieren.

Dies funktioniert jedoch nur mit der Verschiebung der Larmorfrequenz, die nicht
von der Zeit abhängt, die also vom externen Magnetfeld kommt. Die
Spin-Spin-Wechselwirkung kann dadurch nicht rückgängig gemacht werden. Hier
sind wir allerdings auch nur an dieser interessiert, so dass wir den gesuchten
Effekt isoliert haben.

Aus der Stärke des Echos kann dann die homogene transversale Relaxationszeit
$T_2$ bestimmt werden, in dem an den exponenziellen Abfall wieder eine
Exponentialfunktion angepasst wird.

\subsubsection{Carr–Purcell–Sequenz}

Die Carr-Purcell-Sequenz wiederholt die Verzögerungszeit und den $\piup$-Puls
$N$ Male, so dass mehrere Echos zu sehen sind.

Auf diese Weise wird das System in relativ kurzen Abständen wieder in einen
recht definierten Zustand gebracht – Magnetisierung in eine Richtung – und kann
so deutlich länger beobachtet werden. Die Relaxationszeit erhalten wir hier
durch anpassen einer Exponentialfunktion an die Maxima.

\subsubsection{Meiboom–Gill–Sequenz}

Analog zur Carr-Purcell-Sequenz wird hier der zweite Puls wiederholt, jedoch
mit alternierendem Vorzeichen. Auf diese Weise addieren sich Fehler nicht mehr
auf: Sollte der $\piup$-Puls nicht genau die richtige Länge haben, wird der
Fehler, der bei der einen Wiederholung gemacht wird, in der nächsten
korrigiert. Das Ausgangssignal hat zwar auf jedem ungraden Echo einen Fehler,
jedoch sind die Geraden ohne diesen Fehler und höher.

\chapter{Aufbau und Kalibrierung}
\section{Anschließen der Geräte}

Im Aufbau des Experimentes verwenden wir vier Teile:

\begin{itemize}
    \item
        Magnet mit RF-Messkopf
    \item
        PS2 Controller mit Regelung der Magnettemperatur und Steuerung der
        Magnetfeldgradienten
    \item
        Mainframe mit Receiver, Synthesizer, Pulse Programmer, Lock-In /
        Field-Sweep und der Stromversorgung
    \item
        digitales Oszilloskop
\end{itemize}

Zunächst werden \texttt{Pulsed RF-Out} am Synthesizer mit \texttt{Pulsed
RF-In} am Receiver und der Magnet mit \texttt{Sample} am Receiver verbunden.
Nun werden an Pulse Programmer und Synthesizer jeweils \texttt{Q} und
\texttt{I} mit einander verbunden, sowie \texttt{Blanking Out} am Pulse
Programmer mit \texttt{Blanking In} am Receiver. Als nächstes verbinden wir
das Oszilloskop: \texttt{Sync Out} des Pulse Programmers wird an
\texttt{Input Trigger}, \texttt{Env. Out}, \texttt{Q} und \texttt{I} des
Receivers werden an den Kanälen 1, 2 und 3 des Oszilloskops angeschlossen.

Im Anschluss werden die Temperaturregler des PS2 Controllers auf
\texttt{open}, \texttt{TC}, \texttt{Gain} am Receiver
auf \num{0.01} bzw. 75\% und \texttt{Sync} am Pulse Programmer
auf \texttt{A} gestellt.
\texttt{Blanking}, \texttt{Ref Out} und \texttt{Pulse A} an
Receiver, Sythesizer bzw. Pulse Programmer werden an-, \texttt{CW Out} am
Synthesizer und \texttt{MG} und \texttt{Pulse B} am Pulse Programmer werden
ausgeschaltet. 

\section{Temperatureinstellung}

Damit die Temperatur innerhalb des Magneten stabilisiert werden kann, muss
diese zunächst über die Potentiometer auf Raumtemperatur geregelt werden.
Sobald die LED-Anzeige erlischt, ist dies erreicht und der Kippschalter kann
umgelegt werden um die Regelung abzukoppeln.

\section{Tuning des RF-Resonanzkreises}

Wir stellen am Pulse Programmer $A_\text{len}$ auf \SI{2.5}{\micro\second} und
\texttt P auf \SI{100}{\micro\second} und am Oszilloskop den Trigger auf
\texttt{Ext}, \texttt{Normal}, \texttt{Rising} und $> \SI{.1}{\volt}$, den
Sweep auf \SI{2}{\micro\second\per\division} und Kanal 1 auf
\SI{5}{\volt\per\division}, \texttt{DC} und \texttt{full bandwidth}.

Nun wird die Pickup Probe eingesetzt und an Kanal 1 angeschlossen. Die
Tuning-Kondensatoren im Magneten werden nun so eingestellt, dass ein
optimiertes Signal entsteht.

\section{Optimierung des Free Induction Decay Signals}

Die eigentliche Probe wird nun eingesetzt und \texttt{Env Out} des Receivers
wieder an Kanal 1 angeschlossen. Das Signal soll nun möglichst hoch, langsam
und exponentiell abfallend sein. Dazu werden die Magnetfeldgradienten am
PS1~Controller und die Frequenz am Synthesizer verändert. Für ein Feintuning
der Frequenz kann man sich das \texttt{Q}-Signal am Oszilloskop anzeigen
lassen. Die Nachschwingung soll möglichst gering sein.

\section{$\frac\piup2$- und $\piup$-Puls Justage}

Nach der Theorie sollte sich bei Verdopplung von $A_\text{len}$ das Signal
ganz auslöschen. Die Pulslänge soll jetzt so korrigiert werden, dass das
Signal minimal wird. Die so gefundene Zeit ist die Länge des $\piup$- die
Hälfte davon entsprechend die des $\frac{\piup}{2}$-Pulses.

\chapter{Durchführung und Auswertung}

\section{Vorbereitung}

Unser optimiertes Pickup-Signal ist in Abbildung~\ref{fig:pickup}
zu sehen.

\begin{figure}
    \centering
    \includegraphics[width=.8\textwidth]{Opt_Pickup_lang.pdf}
    \includegraphics[width=.8\textwidth]{Opt_Pickup_kurz.pdf}
    \caption{%
    Optimiertes Pickup-Signal
    }
    \label{fig:pickup}
\end{figure}

Wir stellen das Oszilloskop so ein, dass wir das Signal gut sehen können.

Die Gradienten haben wir eingestellt auf $X = \num{0.5}$, $Y = \num{18.0}$, $Z
= \num{32.2}$ und $Z^2 = \num{0.0}$, die gewählte Frequenz ist
\SI{21.16141}{\mega\hertz}. Das entstandene Bild ist in Abbildung~\ref{fig:FID}
zu sehen.

\begin{figure}
    \centering
    \includegraphics[width=\textwidth]{Opt_FID.pdf}
    \caption{%
    Optimiertes FID-Signal
    }
    \label{fig:FID}
\end{figure}

Die Pulsdauer, bei der das Signal minimal ist, erhalten wir bei $A_\text{len}$
= \SI{5.16}{\micro\second}. Die Dauer für das $\piup/2$-Signal ist entsprechend
\SI{2.58}{\micro\second}.

\FloatBarrier
\section{Rabi-Oszillationen}

Unsere Messwerte befinden sich in Tabelle~\ref{tab:rabi}, eine graphische
Darstellung einschließlich Anpassungsfunktionen ist in Abbildung~\ref{fig:rabi}
zu sehen.

\begin{table}
    \centering
    \begin{tabular}{SSSS}
        {$f / \si{\mega\hertz}$} &
        {$A_\text{len} / \si{\micro\second}$} &
        {$Q\text{-Signal} / \si{\volt}$} &
        {$I\text{-Signal} / \si{\volt}$} \\
        \midrule
        %< for f, A_len, gelb_val, gruen_val in rabi_messdaten: ->%
        << f >> & << A_len >> & << gelb_val >> & << gruen_val >> \\
        %< endfor >%
    \end{tabular}
    \caption{%
        Messwerte zur Rabi-Oszillation
    }
    \label{tab:rabi}
\end{table}

\begin{figure}[htbp]
    \centering
    \includegraphics[width=\linewidth]{Rabi.pdf}
    \caption{%
        Messdaten und angepasste Funktionen zur Rabi-Oszillation.
    }
    \label{fig:rabi}
\end{figure}

\FloatBarrier
\section{Longitudinale Relaxationszeit $T_1$}


Da die folgenden Messungen am zweiten Versuchstag durchgeführt wurden,
musste die Frequenz auf $\nu = \SI{21.1573}{\mega\hertz}$ angepasst werden.

\FloatBarrier
\subsection{Sättigungs-Zurückgewinnung}

Die abgelesene Gleichgewichtsmagnetisierung beträgt

\[
    M_0 \hat = \SI{7.525}{\volt}
\]

Unsere Messwerte sind in Tabelle~\ref{tab:saet} einzusehen. In
Abbildung~\ref{fig:saet} ist unsere Messung graphisch dargestellt. Die
Anpassungsfunktion ist

\[
    M(\tau) = M_0\del{1-\exp\del{ -\frac{\tau}{T_1} }}
\]

da wir von einer Anfangsmagnetisierung von \num{0} ausgehen.
Wir erhalten daraus

\begin{align*}
    M_0 &= \SI{<< M_0_sat >>}{\volt}
    \intertext{und}
    T_1 &= \SI{<< T_1_sat >>}{\second}
\end{align*}

\begin{table}[htbp]
    \centering
    \begin{tabular}{S|S}
        {$\tau / \si{\second}$} & {$M / \si{\volt}$} \\
        \midrule
        %< for tau, M_val in Tabelle_Saettigung: ->%
        << tau >> & << M_val >> \\
        %< endfor ->%
    \end{tabular}
    \label{tab:saet}
    \caption{Messwerte zur Bestimmung der longitudinalen Relaxationszeit}
\end{table}

\begin{figure}[htbp]
    \centering
    \includegraphics[width=\linewidth]{Saettigung.pdf}
    \caption{%
        Messdaten und angepasste Funktion zur Bestimmung der longitudinalen
        Relaxationszeit durch die Sättigungsmethode
    }
    \label{fig:saet}
\end{figure}

\FloatBarrier
\subsection{Polarisations-Zurückgewinnung}

Unsere Messwerte sind in Tabelle~\ref{tab:pol} einzusehen, eine graphische
Darstellung in Abbildung~\ref{fig:pol}. Die Anpassungsfunktion ist hier

\[
    M(\tau) = M_0\del{ 1 - 2\exp\del{-\frac{\tau}{T_1}}}
\]

Für die longitudinale Relaxationszeit erhalten wir so

\[
    T_1 = \SI{<< T_1_pol >>}{\second}
\]

Die Gleichgewichtsmagnetisierung ist

\[
    M_0 = \SI{<< M_0_pol >>}{\volt}
\]

\begin{table}[htbp]
    \centering
    \begin{tabular}{S|S}
        {$\tau / \si{\second}$} & {$M / \si{\volt}$} \\
        \midrule
        %< for tau, M_val in Tabelle_Polarisation: ->%
        << tau >> & << M_val >> \\
        %< endfor ->%
    \end{tabular}
    \label{tab:pol}
    \caption{Messwerte zur Bestimmung der longitudinalen Relaxationszeit}
\end{table}

\begin{figure}[htbp]
    \centering
    \includegraphics[width=\linewidth]{Polarisation.pdf}
    \caption{%
        Messdaten und angepasste Funktion zur Bestimmung der longitudinalen
        Relaxationszeit durch die Polarisationsmethode
    }
    \label{fig:pol}
\end{figure}

\FloatBarrier
\section{Effektive Transversale Relaxationszeit $T_2^*$}

Unsere Messwerte sind in Tabelle~\ref{tab:eff} und Abbildung~\ref{fig:eff}
gezeigt. Aus der Anpassungsfunktion

\begin{align*}
    M(t) = M_0\exp\del{-\frac{t}{T_2^*}}+M_\text{Offset}
    \intertext{folgt für die effektive transversale Relaxationszeit:}
    T_2^* &= \SI{<< T_eff >>}{\second}
\end{align*}

\begin{table}
    \centering
    \small
    \tabcolsep=0.11cm
    \begin{tabular}{SS|SS|SS|SS|SS|SS}
        {$t / \si{\second}$} & {$M / \si{\volt}$} &
        {$t / \si{\second}$} & {$M / \si{\volt}$} &
        {$t / \si{\second}$} & {$M / \si{\volt}$} &
        {$t / \si{\second}$} & {$M / \si{\volt}$} &
        {$t / \si{\second}$} & {$M / \si{\volt}$} &
        {$t / \si{\second}$} & {$M / \si{\volt}$} \\
        \midrule
        %< for t1, M1, t2, M2, t3, M3, t4, M4, t5, M5, t6, M6 in Tabelle_Effektiv: ->%
        << t1 >> & << M1 >> &
        << t2 >> & << M2 >> &
        << t3 >> & << M3 >> &
        << t4 >> & << M4 >> &
        << t5 >> & << M5 >> &
        << t6 >> & << M6 >>  \\
        %< endfor ->%
    \end{tabular}
    \caption{Messdaten zur effektiven transversalen Relaxationszeit}
    \label{tab:eff}
\end{table}


\begin{figure}[htbp]
    \centering
    \includegraphics[width=\linewidth]{effektiv.pdf}
    \caption{%
    }
    \label{fig:eff}
\end{figure}

\FloatBarrier
\subsection{Homogene Transversale Relaxationszeit $T_2$}

\FloatBarrier
\subsubsection{Hahn–Spinecho–Sequenz}

Unsere Messergebnisse sind in Tabelle~\ref{tab:hahn} und
Abbildung~\ref{fig:hahn} zu sehen. Aus der angepassten Funktion

\begin{align*}
    M(\tau) &= M_0\exp\del{-\frac{\tau}{T_2}}
    \intertext{erhalten wir für die homogene transversale Relaxationszeit}
    T_2 &= \SI{<< T_2_hahn >>}{\second}
\end{align*}

\begin{table}
    \centering
    \begin{tabular}{S|S}
        {$\tau / \si{\second}$} & {$M / \si{\volt}$} \\
        \midrule
        %< for tau, M_val in Tabelle_Hahn: ->%
        << tau >> & << M_val >> \\
        %< endfor ->%
    \end{tabular}
    \caption{%
        Messwerte zur Hahn-Spinecho-Sequenz
    }
    \label{tab:hahn}
\end{table}

\begin{figure}[htbp]
    \centering
    \includegraphics[width=\linewidth]{Hahn.pdf}
    \caption{%
    }
    \label{fig:hahn}
\end{figure}

\FloatBarrier
\subsubsection{Carr–Purcell–Sequenz}

$\tau = \SI{5.00}{\micro\second}$
$A_\text{len} = \SI{2.66}{\micro\second}$
$B_\text{len} = \SI{5.32}{\micro\second}$
$P = \SI{580}{\milli\second}$
$N = \num{20}$

\FloatBarrier
\subsubsection{Meiboom–Gill–Sequenz}

%%%%%%%%%%%%%%%%%%%%%%%%%%%%%%%%%%%%%%%%%%%%%%%%%%%%%%%%%%%%%%%%%%%%%%%%%%%%%%%
%                               Zusammenfassung                               %
%%%%%%%%%%%%%%%%%%%%%%%%%%%%%%%%%%%%%%%%%%%%%%%%%%%%%%%%%%%%%%%%%%%%%%%%%%%%%%%

\FloatBarrier
\chapter{Zusammenfassung}

\fehlt

%%%%%%%%%%%%%%%%%%%%%%%%%%%%%%%%%%%%%%%%%%%%%%%%%%%%%%%%%%%%%%%%%%%%%%%%%%%%%%%
%                                   Anhang                                    %
%%%%%%%%%%%%%%%%%%%%%%%%%%%%%%%%%%%%%%%%%%%%%%%%%%%%%%%%%%%%%%%%%%%%%%%%%%%%%%%

\FloatBarrier
\begin{appendix}
    \FloatBarrier
    \chapter{\LaTeX-Quelltext}

    Der \LaTeX-Quelltext zu allen Protokollen in diesem Praktikum kann auf
    \ref{it:mu} eingesehen werden. Die Quellen für alle Protokolle in diesem
    Praktikum können auf \ref{it:github/alles} eingesehen werden. Die
    \LaTeX-Datei wird aus \ref{it:github/template} generiert.

    \begin{enumerate}
        \item
            \label{it:mu}
            \url{http://martin-ueding.de/de/university/physik412/}
        \item
            \label{it:github/alles}
            \url{https://github.com/martin-ueding/physik412-Protokolle/}
        \item
            \label{it:github/template}
            \url{https://github.com/martin-ueding/physik412-Protokolle/blob/master/\versuchsnummer/Template.tex}
    \end{enumerate}
\end{appendix}

%%%%%%%%%%%%%%%%%%%%%%%%%%%%%%%%%%%%%%%%%%%%%%%%%%%%%%%%%%%%%%%%%%%%%%%%%%%%%%%
%                                  Literatur                                  %
%%%%%%%%%%%%%%%%%%%%%%%%%%%%%%%%%%%%%%%%%%%%%%%%%%%%%%%%%%%%%%%%%%%%%%%%%%%%%%%

\FloatBarrier
\printbibliography

\end{document}

% vim: et spell spelllang=de tw=79
