% Copyright © 2013 Martin Ueding <dev@martin-ueding.de>

% Copyright © 2012-2013 Martin Ueding <dev@martin-ueding.de>

% This is my general purpose LaTeX header file for writing German documents.
% Ideally, you include this using a simple ``% Copyright © 2012-2013 Martin Ueding <dev@martin-ueding.de>

% This is my general purpose LaTeX header file for writing German documents.
% Ideally, you include this using a simple ``% Copyright © 2012-2013 Martin Ueding <dev@martin-ueding.de>

% This is my general purpose LaTeX header file for writing German documents.
% Ideally, you include this using a simple ``\input{header.tex}`` in your main
% document and start with ``\title`` and ``\begin{document}`` afterwards.

% If you need to add additional packages, I recommend not doing this in this
% file, but in your main document. That way, you can just drop in a new
% ``header.tex`` and get all the new commands without having to merge manually.

% Since this file encorporates a CC-BY-SA fragment, this whole files is
% licensed under the CC-BY-SA license.

\documentclass[11pt, ngerman, fleqn, DIV=15, BCOR=2cm, headinclude]{scrartcl}

\usepackage{graphicx}

% Environment to quote the problem. Currently, this is just a new name for the
% quote environment.
\newenvironment{problem}{\begin{quote}\textsf{\textbf{Aufgabenstellung:}}\quad}{\end{quote}}

\setkomafont{caption}{\sf}
\setkomafont{captionlabel}{\usekomafont{caption}}

%%%%%%%%%%%%%%%%%%%%%%%%%%%%%%%%%%%%%%%%%%%%%%%%%%%%%%%%%%%%%%%%%%%%%%%%%%%%%%%
%                                Locale, date                                 %
%%%%%%%%%%%%%%%%%%%%%%%%%%%%%%%%%%%%%%%%%%%%%%%%%%%%%%%%%%%%%%%%%%%%%%%%%%%%%%%

\usepackage{babel}
\usepackage[iso]{isodate}

%%%%%%%%%%%%%%%%%%%%%%%%%%%%%%%%%%%%%%%%%%%%%%%%%%%%%%%%%%%%%%%%%%%%%%%%%%%%%%%
%                          Margins and other spacing                          %
%%%%%%%%%%%%%%%%%%%%%%%%%%%%%%%%%%%%%%%%%%%%%%%%%%%%%%%%%%%%%%%%%%%%%%%%%%%%%%%

\usepackage[parfill]{parskip}
\usepackage{setspace}
\usepackage[activate]{microtype}

\setlength{\columnsep}{2cm}

%%%%%%%%%%%%%%%%%%%%%%%%%%%%%%%%%%%%%%%%%%%%%%%%%%%%%%%%%%%%%%%%%%%%%%%%%%%%%%%
%                                    Color                                    %
%%%%%%%%%%%%%%%%%%%%%%%%%%%%%%%%%%%%%%%%%%%%%%%%%%%%%%%%%%%%%%%%%%%%%%%%%%%%%%%

\usepackage[usenames, dvipsnames]{xcolor}

\colorlet{darkred}{red!70!black}
\colorlet{darkblue}{blue!70!black}
\colorlet{darkgreen}{green!40!black}

%%%%%%%%%%%%%%%%%%%%%%%%%%%%%%%%%%%%%%%%%%%%%%%%%%%%%%%%%%%%%%%%%%%%%%%%%%%%%%%
%                         Font and font like settings                         %
%%%%%%%%%%%%%%%%%%%%%%%%%%%%%%%%%%%%%%%%%%%%%%%%%%%%%%%%%%%%%%%%%%%%%%%%%%%%%%%

% This replaces all fonts with Bitstream Charter, Bitstream Vera Sans and
% Bitstream Vera Mono. Math will be rendered in Charter.
\usepackage[charter, greekuppercase=italicized]{mathdesign}
\usepackage{beramono}
\usepackage{berasans}

% Bold, sans-serif tensors. This fragment is taken from “egreg” from
% http://tex.stackexchange.com/a/82747/8945 and licensed under `CC-BY-SA
% <https://creativecommons.org/licenses/by-sa/3.0/>`_.
\usepackage{bm}
\DeclareMathAlphabet{\mathsfit}{\encodingdefault}{\sfdefault}{m}{sl}
\SetMathAlphabet{\mathsfit}{bold}{\encodingdefault}{\sfdefault}{bx}{sl}
\newcommand{\tens}[1]{\bm{\mathsfit{#1}}}

% Bold vectors.
\renewcommand{\vec}[1]{\boldsymbol{#1}}

%%%%%%%%%%%%%%%%%%%%%%%%%%%%%%%%%%%%%%%%%%%%%%%%%%%%%%%%%%%%%%%%%%%%%%%%%%%%%%%
%                               Input encoding                                %
%%%%%%%%%%%%%%%%%%%%%%%%%%%%%%%%%%%%%%%%%%%%%%%%%%%%%%%%%%%%%%%%%%%%%%%%%%%%%%%

\usepackage[T1]{fontenc}
\usepackage[utf8]{inputenc}

%%%%%%%%%%%%%%%%%%%%%%%%%%%%%%%%%%%%%%%%%%%%%%%%%%%%%%%%%%%%%%%%%%%%%%%%%%%%%%%
%                         Hyperrefs and PDF metadata                          %
%%%%%%%%%%%%%%%%%%%%%%%%%%%%%%%%%%%%%%%%%%%%%%%%%%%%%%%%%%%%%%%%%%%%%%%%%%%%%%%

\usepackage{hyperref}
\usepackage{lastpage}

% This sets the author in the properties of the PDF as well. If you want to
% change it, just override it with another ``\hypersetup`` call.
\hypersetup{
	breaklinks=false,
	citecolor=darkgreen,
	colorlinks=true,
	linkcolor=darkblue,
	menucolor=black,
	pdfauthor={Martin Ueding},
	urlcolor=darkblue,
}

%%%%%%%%%%%%%%%%%%%%%%%%%%%%%%%%%%%%%%%%%%%%%%%%%%%%%%%%%%%%%%%%%%%%%%%%%%%%%%%
%                               Math Operators                                %
%%%%%%%%%%%%%%%%%%%%%%%%%%%%%%%%%%%%%%%%%%%%%%%%%%%%%%%%%%%%%%%%%%%%%%%%%%%%%%%

% AMS environments like ``align`` and theorems like ``proof``.
\usepackage{amsmath}
\usepackage{amsthm}

% Common math constructs like partial derivatives.
\usepackage{commath}

% Physical units.
\usepackage[output-decimal-marker={,}]{siunitx}

% Since I use mathdesign with italic uppercase greek characters, the Ohm unit will be displayed with an italic Ω by default. Units have to be roman, so this forces it the right way.
\DeclareSIUnit{\ohm}{$\Omegaup$}
\DeclareSIUnit{\division}{DIV}
\DeclareSIUnit{\voltss}{$\mathrm{V_{SS}}$}

% Word like operators.
\DeclareMathOperator{\acosh}{arcosh}
\DeclareMathOperator{\arcosh}{arcosh}
\DeclareMathOperator{\arcsinh}{arsinh}
\DeclareMathOperator{\arsinh}{arsinh}
\DeclareMathOperator{\asinh}{arsinh}
\DeclareMathOperator{\card}{card}
\DeclareMathOperator{\csch}{cshs}
\DeclareMathOperator{\diam}{diam}
\DeclareMathOperator{\sech}{sech}
\renewcommand{\Im}{\mathop{{}\mathrm{Im}}\nolimits}
\renewcommand{\Re}{\mathop{{}\mathrm{Re}}\nolimits}

% Fourier transform.
\DeclareMathOperator{\fourier}{\ensuremath{\mathcal{F}}}

% Roman versions of “e” and “i” to serve as Euler's number and the imaginary
% constant.
\newcommand{\ee}{\eup}
\newcommand{\eup}{\mathrm e}
\newcommand{\ii}{\iup}
\newcommand{\iup}{\mathrm i}

% Symbols for the various mathematical fields (natural numbers, integers,
% rational numbers, real numbers, complex numbers).
\newcommand{\C}{\ensuremath{\mathbb C}}
\newcommand{\N}{\ensuremath{\mathbb N}}
\newcommand{\Q}{\ensuremath{\mathbb Q}}
\newcommand{\R}{\ensuremath{\mathbb R}}
\newcommand{\Z}{\ensuremath{\mathbb Z}}

% Shape like operators.
\DeclareMathOperator{\dalambert}{\Box}
\DeclareMathOperator{\laplace}{\bigtriangleup}
\newcommand{\curl}{\vnabla \times}
\newcommand{\divergence}[1]{\inner{\vnabla}{#1}}
\newcommand{\vnabla}{\vec \nabla}

\newcommand{\half}{\frac 12}

% Unit vector (German „Einheitsvektor“).
\newcommand{\ev}{\hat{\vec e}}

% Scientific notation for large numbers.
\newcommand{\e}[1]{\cdot 10^{#1}}

% Mathematician's notation for the inner (scalar, dot) product.
\newcommand{\bracket}[1]{\left\langle #1 \right\rangle}
\newcommand{\inner}[2]{\bracket{#1, #2}}

% Placeholders.
\newcommand{\emesswert}{\del{\messwert \pm \messwert}}
\newcommand{\fehlt}{\textcolor{darkred}{Hier fehlen noch Inhalte.}}
\newcommand{\messwert}{\textcolor{blue}{\square}}
\newcommand{\punkte}{\phantom{xxxxx}}
\newcommand{\punktevon}[1]{\begin{flushright}/ #1\end{flushright}}

% Separator for equations on a single line.
\newcommand{\eqnsep}{,\quad}

% Quantum Mechanics
\usepackage{braket}

%%%%%%%%%%%%%%%%%%%%%%%%%%%%%%%%%%%%%%%%%%%%%%%%%%%%%%%%%%%%%%%%%%%%%%%%%%%%%%%
%                                  Headings                                   %
%%%%%%%%%%%%%%%%%%%%%%%%%%%%%%%%%%%%%%%%%%%%%%%%%%%%%%%%%%%%%%%%%%%%%%%%%%%%%%%

% This will set fancy headings to the top of the page. The page number will be
% accompanied by the total number of pages. That way, you will know if any page
% is missing.
%
% If you do not want this for your document, you can just use
% ``\pagestyle{plain}``.

\usepackage{scrpage2}

\pagestyle{scrheadings}
\automark{section}
\cfoot{\footnotesize{Seite \thepage\ / \pageref{LastPage}}}
\chead{}
\ihead{}
\ohead{\rightmark}
\setheadsepline{.4pt}

%%%%%%%%%%%%%%%%%%%%%%%%%%%%%%%%%%%%%%%%%%%%%%%%%%%%%%%%%%%%%%%%%%%%%%%%%%%%%%%
%                            Bibliography (BibTeX)                            %
%%%%%%%%%%%%%%%%%%%%%%%%%%%%%%%%%%%%%%%%%%%%%%%%%%%%%%%%%%%%%%%%%%%%%%%%%%%%%%%

\newcommand{\bibliographyfile}{../../central-bibtex/Central}
\bibliographystyle{apalike2}

%%%%%%%%%%%%%%%%%%%%%%%%%%%%%%%%%%%%%%%%%%%%%%%%%%%%%%%%%%%%%%%%%%%%%%%%%%%%%%%
%                                Abbreviations                                %
%%%%%%%%%%%%%%%%%%%%%%%%%%%%%%%%%%%%%%%%%%%%%%%%%%%%%%%%%%%%%%%%%%%%%%%%%%%%%%%

\newcommand{\dhabk}{\mbox{d.\,h.}}

%%%%%%%%%%%%%%%%%%%%%%%%%%%%%%%%%%%%%%%%%%%%%%%%%%%%%%%%%%%%%%%%%%%%%%%%%%%%%%%
%                                  Licences                                   %
%%%%%%%%%%%%%%%%%%%%%%%%%%%%%%%%%%%%%%%%%%%%%%%%%%%%%%%%%%%%%%%%%%%%%%%%%%%%%%%

\usepackage{ccicons}

\newcommand{\ccbysadetext}{%
	\begin{small}
		Dieses Werk bzw. Inhalt steht unter einer
		\href{http://creativecommons.org/licenses/by-sa/3.0/deed.de}{%
			Creative Commons Namensnennung - Weitergabe unter gleichen
		Bedingungen 3.0 Unported Lizenz}.
	\end{small}
}

\newcommand{\ccbysadetitle}{%
	Lizenz: \href{http://creativecommons.org/licenses/by-sa/3.0/deed.de}
	{CC-BY-SA 3.0 \ccbysa}
}
`` in your main
% document and start with ``\title`` and ``\begin{document}`` afterwards.

% If you need to add additional packages, I recommend not doing this in this
% file, but in your main document. That way, you can just drop in a new
% ``header.tex`` and get all the new commands without having to merge manually.

% Since this file encorporates a CC-BY-SA fragment, this whole files is
% licensed under the CC-BY-SA license.

\documentclass[11pt, ngerman, fleqn, DIV=15, BCOR=2cm, headinclude]{scrartcl}

\usepackage{graphicx}

% Environment to quote the problem. Currently, this is just a new name for the
% quote environment.
\newenvironment{problem}{\begin{quote}\textsf{\textbf{Aufgabenstellung:}}\quad}{\end{quote}}

\setkomafont{caption}{\sf}
\setkomafont{captionlabel}{\usekomafont{caption}}

%%%%%%%%%%%%%%%%%%%%%%%%%%%%%%%%%%%%%%%%%%%%%%%%%%%%%%%%%%%%%%%%%%%%%%%%%%%%%%%
%                                Locale, date                                 %
%%%%%%%%%%%%%%%%%%%%%%%%%%%%%%%%%%%%%%%%%%%%%%%%%%%%%%%%%%%%%%%%%%%%%%%%%%%%%%%

\usepackage{babel}
\usepackage[iso]{isodate}

%%%%%%%%%%%%%%%%%%%%%%%%%%%%%%%%%%%%%%%%%%%%%%%%%%%%%%%%%%%%%%%%%%%%%%%%%%%%%%%
%                          Margins and other spacing                          %
%%%%%%%%%%%%%%%%%%%%%%%%%%%%%%%%%%%%%%%%%%%%%%%%%%%%%%%%%%%%%%%%%%%%%%%%%%%%%%%

\usepackage[parfill]{parskip}
\usepackage{setspace}
\usepackage[activate]{microtype}

\setlength{\columnsep}{2cm}

%%%%%%%%%%%%%%%%%%%%%%%%%%%%%%%%%%%%%%%%%%%%%%%%%%%%%%%%%%%%%%%%%%%%%%%%%%%%%%%
%                                    Color                                    %
%%%%%%%%%%%%%%%%%%%%%%%%%%%%%%%%%%%%%%%%%%%%%%%%%%%%%%%%%%%%%%%%%%%%%%%%%%%%%%%

\usepackage[usenames, dvipsnames]{xcolor}

\colorlet{darkred}{red!70!black}
\colorlet{darkblue}{blue!70!black}
\colorlet{darkgreen}{green!40!black}

%%%%%%%%%%%%%%%%%%%%%%%%%%%%%%%%%%%%%%%%%%%%%%%%%%%%%%%%%%%%%%%%%%%%%%%%%%%%%%%
%                         Font and font like settings                         %
%%%%%%%%%%%%%%%%%%%%%%%%%%%%%%%%%%%%%%%%%%%%%%%%%%%%%%%%%%%%%%%%%%%%%%%%%%%%%%%

% This replaces all fonts with Bitstream Charter, Bitstream Vera Sans and
% Bitstream Vera Mono. Math will be rendered in Charter.
\usepackage[charter, greekuppercase=italicized]{mathdesign}
\usepackage{beramono}
\usepackage{berasans}

% Bold, sans-serif tensors. This fragment is taken from “egreg” from
% http://tex.stackexchange.com/a/82747/8945 and licensed under `CC-BY-SA
% <https://creativecommons.org/licenses/by-sa/3.0/>`_.
\usepackage{bm}
\DeclareMathAlphabet{\mathsfit}{\encodingdefault}{\sfdefault}{m}{sl}
\SetMathAlphabet{\mathsfit}{bold}{\encodingdefault}{\sfdefault}{bx}{sl}
\newcommand{\tens}[1]{\bm{\mathsfit{#1}}}

% Bold vectors.
\renewcommand{\vec}[1]{\boldsymbol{#1}}

%%%%%%%%%%%%%%%%%%%%%%%%%%%%%%%%%%%%%%%%%%%%%%%%%%%%%%%%%%%%%%%%%%%%%%%%%%%%%%%
%                               Input encoding                                %
%%%%%%%%%%%%%%%%%%%%%%%%%%%%%%%%%%%%%%%%%%%%%%%%%%%%%%%%%%%%%%%%%%%%%%%%%%%%%%%

\usepackage[T1]{fontenc}
\usepackage[utf8]{inputenc}

%%%%%%%%%%%%%%%%%%%%%%%%%%%%%%%%%%%%%%%%%%%%%%%%%%%%%%%%%%%%%%%%%%%%%%%%%%%%%%%
%                         Hyperrefs and PDF metadata                          %
%%%%%%%%%%%%%%%%%%%%%%%%%%%%%%%%%%%%%%%%%%%%%%%%%%%%%%%%%%%%%%%%%%%%%%%%%%%%%%%

\usepackage{hyperref}
\usepackage{lastpage}

% This sets the author in the properties of the PDF as well. If you want to
% change it, just override it with another ``\hypersetup`` call.
\hypersetup{
	breaklinks=false,
	citecolor=darkgreen,
	colorlinks=true,
	linkcolor=darkblue,
	menucolor=black,
	pdfauthor={Martin Ueding},
	urlcolor=darkblue,
}

%%%%%%%%%%%%%%%%%%%%%%%%%%%%%%%%%%%%%%%%%%%%%%%%%%%%%%%%%%%%%%%%%%%%%%%%%%%%%%%
%                               Math Operators                                %
%%%%%%%%%%%%%%%%%%%%%%%%%%%%%%%%%%%%%%%%%%%%%%%%%%%%%%%%%%%%%%%%%%%%%%%%%%%%%%%

% AMS environments like ``align`` and theorems like ``proof``.
\usepackage{amsmath}
\usepackage{amsthm}

% Common math constructs like partial derivatives.
\usepackage{commath}

% Physical units.
\usepackage[output-decimal-marker={,}]{siunitx}

% Since I use mathdesign with italic uppercase greek characters, the Ohm unit will be displayed with an italic Ω by default. Units have to be roman, so this forces it the right way.
\DeclareSIUnit{\ohm}{$\Omegaup$}
\DeclareSIUnit{\division}{DIV}
\DeclareSIUnit{\voltss}{$\mathrm{V_{SS}}$}

% Word like operators.
\DeclareMathOperator{\acosh}{arcosh}
\DeclareMathOperator{\arcosh}{arcosh}
\DeclareMathOperator{\arcsinh}{arsinh}
\DeclareMathOperator{\arsinh}{arsinh}
\DeclareMathOperator{\asinh}{arsinh}
\DeclareMathOperator{\card}{card}
\DeclareMathOperator{\csch}{cshs}
\DeclareMathOperator{\diam}{diam}
\DeclareMathOperator{\sech}{sech}
\renewcommand{\Im}{\mathop{{}\mathrm{Im}}\nolimits}
\renewcommand{\Re}{\mathop{{}\mathrm{Re}}\nolimits}

% Fourier transform.
\DeclareMathOperator{\fourier}{\ensuremath{\mathcal{F}}}

% Roman versions of “e” and “i” to serve as Euler's number and the imaginary
% constant.
\newcommand{\ee}{\eup}
\newcommand{\eup}{\mathrm e}
\newcommand{\ii}{\iup}
\newcommand{\iup}{\mathrm i}

% Symbols for the various mathematical fields (natural numbers, integers,
% rational numbers, real numbers, complex numbers).
\newcommand{\C}{\ensuremath{\mathbb C}}
\newcommand{\N}{\ensuremath{\mathbb N}}
\newcommand{\Q}{\ensuremath{\mathbb Q}}
\newcommand{\R}{\ensuremath{\mathbb R}}
\newcommand{\Z}{\ensuremath{\mathbb Z}}

% Shape like operators.
\DeclareMathOperator{\dalambert}{\Box}
\DeclareMathOperator{\laplace}{\bigtriangleup}
\newcommand{\curl}{\vnabla \times}
\newcommand{\divergence}[1]{\inner{\vnabla}{#1}}
\newcommand{\vnabla}{\vec \nabla}

\newcommand{\half}{\frac 12}

% Unit vector (German „Einheitsvektor“).
\newcommand{\ev}{\hat{\vec e}}

% Scientific notation for large numbers.
\newcommand{\e}[1]{\cdot 10^{#1}}

% Mathematician's notation for the inner (scalar, dot) product.
\newcommand{\bracket}[1]{\left\langle #1 \right\rangle}
\newcommand{\inner}[2]{\bracket{#1, #2}}

% Placeholders.
\newcommand{\emesswert}{\del{\messwert \pm \messwert}}
\newcommand{\fehlt}{\textcolor{darkred}{Hier fehlen noch Inhalte.}}
\newcommand{\messwert}{\textcolor{blue}{\square}}
\newcommand{\punkte}{\phantom{xxxxx}}
\newcommand{\punktevon}[1]{\begin{flushright}/ #1\end{flushright}}

% Separator for equations on a single line.
\newcommand{\eqnsep}{,\quad}

% Quantum Mechanics
\usepackage{braket}

%%%%%%%%%%%%%%%%%%%%%%%%%%%%%%%%%%%%%%%%%%%%%%%%%%%%%%%%%%%%%%%%%%%%%%%%%%%%%%%
%                                  Headings                                   %
%%%%%%%%%%%%%%%%%%%%%%%%%%%%%%%%%%%%%%%%%%%%%%%%%%%%%%%%%%%%%%%%%%%%%%%%%%%%%%%

% This will set fancy headings to the top of the page. The page number will be
% accompanied by the total number of pages. That way, you will know if any page
% is missing.
%
% If you do not want this for your document, you can just use
% ``\pagestyle{plain}``.

\usepackage{scrpage2}

\pagestyle{scrheadings}
\automark{section}
\cfoot{\footnotesize{Seite \thepage\ / \pageref{LastPage}}}
\chead{}
\ihead{}
\ohead{\rightmark}
\setheadsepline{.4pt}

%%%%%%%%%%%%%%%%%%%%%%%%%%%%%%%%%%%%%%%%%%%%%%%%%%%%%%%%%%%%%%%%%%%%%%%%%%%%%%%
%                            Bibliography (BibTeX)                            %
%%%%%%%%%%%%%%%%%%%%%%%%%%%%%%%%%%%%%%%%%%%%%%%%%%%%%%%%%%%%%%%%%%%%%%%%%%%%%%%

\newcommand{\bibliographyfile}{../../central-bibtex/Central}
\bibliographystyle{apalike2}

%%%%%%%%%%%%%%%%%%%%%%%%%%%%%%%%%%%%%%%%%%%%%%%%%%%%%%%%%%%%%%%%%%%%%%%%%%%%%%%
%                                Abbreviations                                %
%%%%%%%%%%%%%%%%%%%%%%%%%%%%%%%%%%%%%%%%%%%%%%%%%%%%%%%%%%%%%%%%%%%%%%%%%%%%%%%

\newcommand{\dhabk}{\mbox{d.\,h.}}

%%%%%%%%%%%%%%%%%%%%%%%%%%%%%%%%%%%%%%%%%%%%%%%%%%%%%%%%%%%%%%%%%%%%%%%%%%%%%%%
%                                  Licences                                   %
%%%%%%%%%%%%%%%%%%%%%%%%%%%%%%%%%%%%%%%%%%%%%%%%%%%%%%%%%%%%%%%%%%%%%%%%%%%%%%%

\usepackage{ccicons}

\newcommand{\ccbysadetext}{%
	\begin{small}
		Dieses Werk bzw. Inhalt steht unter einer
		\href{http://creativecommons.org/licenses/by-sa/3.0/deed.de}{%
			Creative Commons Namensnennung - Weitergabe unter gleichen
		Bedingungen 3.0 Unported Lizenz}.
	\end{small}
}

\newcommand{\ccbysadetitle}{%
	Lizenz: \href{http://creativecommons.org/licenses/by-sa/3.0/deed.de}
	{CC-BY-SA 3.0 \ccbysa}
}
`` in your main
% document and start with ``\title`` and ``\begin{document}`` afterwards.

% If you need to add additional packages, I recommend not doing this in this
% file, but in your main document. That way, you can just drop in a new
% ``header.tex`` and get all the new commands without having to merge manually.

% Since this file encorporates a CC-BY-SA fragment, this whole files is
% licensed under the CC-BY-SA license.

\documentclass[11pt, ngerman, fleqn, DIV=15, BCOR=2cm, headinclude]{scrartcl}

\usepackage{graphicx}

% Environment to quote the problem. Currently, this is just a new name for the
% quote environment.
\newenvironment{problem}{\begin{quote}\textsf{\textbf{Aufgabenstellung:}}\quad}{\end{quote}}

\setkomafont{caption}{\sf}
\setkomafont{captionlabel}{\usekomafont{caption}}

%%%%%%%%%%%%%%%%%%%%%%%%%%%%%%%%%%%%%%%%%%%%%%%%%%%%%%%%%%%%%%%%%%%%%%%%%%%%%%%
%                                Locale, date                                 %
%%%%%%%%%%%%%%%%%%%%%%%%%%%%%%%%%%%%%%%%%%%%%%%%%%%%%%%%%%%%%%%%%%%%%%%%%%%%%%%

\usepackage{babel}
\usepackage[iso]{isodate}

%%%%%%%%%%%%%%%%%%%%%%%%%%%%%%%%%%%%%%%%%%%%%%%%%%%%%%%%%%%%%%%%%%%%%%%%%%%%%%%
%                          Margins and other spacing                          %
%%%%%%%%%%%%%%%%%%%%%%%%%%%%%%%%%%%%%%%%%%%%%%%%%%%%%%%%%%%%%%%%%%%%%%%%%%%%%%%

\usepackage[parfill]{parskip}
\usepackage{setspace}
\usepackage[activate]{microtype}

\setlength{\columnsep}{2cm}

%%%%%%%%%%%%%%%%%%%%%%%%%%%%%%%%%%%%%%%%%%%%%%%%%%%%%%%%%%%%%%%%%%%%%%%%%%%%%%%
%                                    Color                                    %
%%%%%%%%%%%%%%%%%%%%%%%%%%%%%%%%%%%%%%%%%%%%%%%%%%%%%%%%%%%%%%%%%%%%%%%%%%%%%%%

\usepackage[usenames, dvipsnames]{xcolor}

\colorlet{darkred}{red!70!black}
\colorlet{darkblue}{blue!70!black}
\colorlet{darkgreen}{green!40!black}

%%%%%%%%%%%%%%%%%%%%%%%%%%%%%%%%%%%%%%%%%%%%%%%%%%%%%%%%%%%%%%%%%%%%%%%%%%%%%%%
%                         Font and font like settings                         %
%%%%%%%%%%%%%%%%%%%%%%%%%%%%%%%%%%%%%%%%%%%%%%%%%%%%%%%%%%%%%%%%%%%%%%%%%%%%%%%

% This replaces all fonts with Bitstream Charter, Bitstream Vera Sans and
% Bitstream Vera Mono. Math will be rendered in Charter.
\usepackage[charter, greekuppercase=italicized]{mathdesign}
\usepackage{beramono}
\usepackage{berasans}

% Bold, sans-serif tensors. This fragment is taken from “egreg” from
% http://tex.stackexchange.com/a/82747/8945 and licensed under `CC-BY-SA
% <https://creativecommons.org/licenses/by-sa/3.0/>`_.
\usepackage{bm}
\DeclareMathAlphabet{\mathsfit}{\encodingdefault}{\sfdefault}{m}{sl}
\SetMathAlphabet{\mathsfit}{bold}{\encodingdefault}{\sfdefault}{bx}{sl}
\newcommand{\tens}[1]{\bm{\mathsfit{#1}}}

% Bold vectors.
\renewcommand{\vec}[1]{\boldsymbol{#1}}

%%%%%%%%%%%%%%%%%%%%%%%%%%%%%%%%%%%%%%%%%%%%%%%%%%%%%%%%%%%%%%%%%%%%%%%%%%%%%%%
%                               Input encoding                                %
%%%%%%%%%%%%%%%%%%%%%%%%%%%%%%%%%%%%%%%%%%%%%%%%%%%%%%%%%%%%%%%%%%%%%%%%%%%%%%%

\usepackage[T1]{fontenc}
\usepackage[utf8]{inputenc}

%%%%%%%%%%%%%%%%%%%%%%%%%%%%%%%%%%%%%%%%%%%%%%%%%%%%%%%%%%%%%%%%%%%%%%%%%%%%%%%
%                         Hyperrefs and PDF metadata                          %
%%%%%%%%%%%%%%%%%%%%%%%%%%%%%%%%%%%%%%%%%%%%%%%%%%%%%%%%%%%%%%%%%%%%%%%%%%%%%%%

\usepackage{hyperref}
\usepackage{lastpage}

% This sets the author in the properties of the PDF as well. If you want to
% change it, just override it with another ``\hypersetup`` call.
\hypersetup{
	breaklinks=false,
	citecolor=darkgreen,
	colorlinks=true,
	linkcolor=darkblue,
	menucolor=black,
	pdfauthor={Martin Ueding},
	urlcolor=darkblue,
}

%%%%%%%%%%%%%%%%%%%%%%%%%%%%%%%%%%%%%%%%%%%%%%%%%%%%%%%%%%%%%%%%%%%%%%%%%%%%%%%
%                               Math Operators                                %
%%%%%%%%%%%%%%%%%%%%%%%%%%%%%%%%%%%%%%%%%%%%%%%%%%%%%%%%%%%%%%%%%%%%%%%%%%%%%%%

% AMS environments like ``align`` and theorems like ``proof``.
\usepackage{amsmath}
\usepackage{amsthm}

% Common math constructs like partial derivatives.
\usepackage{commath}

% Physical units.
\usepackage[output-decimal-marker={,}]{siunitx}

% Since I use mathdesign with italic uppercase greek characters, the Ohm unit will be displayed with an italic Ω by default. Units have to be roman, so this forces it the right way.
\DeclareSIUnit{\ohm}{$\Omegaup$}
\DeclareSIUnit{\division}{DIV}
\DeclareSIUnit{\voltss}{$\mathrm{V_{SS}}$}

% Word like operators.
\DeclareMathOperator{\acosh}{arcosh}
\DeclareMathOperator{\arcosh}{arcosh}
\DeclareMathOperator{\arcsinh}{arsinh}
\DeclareMathOperator{\arsinh}{arsinh}
\DeclareMathOperator{\asinh}{arsinh}
\DeclareMathOperator{\card}{card}
\DeclareMathOperator{\csch}{cshs}
\DeclareMathOperator{\diam}{diam}
\DeclareMathOperator{\sech}{sech}
\renewcommand{\Im}{\mathop{{}\mathrm{Im}}\nolimits}
\renewcommand{\Re}{\mathop{{}\mathrm{Re}}\nolimits}

% Fourier transform.
\DeclareMathOperator{\fourier}{\ensuremath{\mathcal{F}}}

% Roman versions of “e” and “i” to serve as Euler's number and the imaginary
% constant.
\newcommand{\ee}{\eup}
\newcommand{\eup}{\mathrm e}
\newcommand{\ii}{\iup}
\newcommand{\iup}{\mathrm i}

% Symbols for the various mathematical fields (natural numbers, integers,
% rational numbers, real numbers, complex numbers).
\newcommand{\C}{\ensuremath{\mathbb C}}
\newcommand{\N}{\ensuremath{\mathbb N}}
\newcommand{\Q}{\ensuremath{\mathbb Q}}
\newcommand{\R}{\ensuremath{\mathbb R}}
\newcommand{\Z}{\ensuremath{\mathbb Z}}

% Shape like operators.
\DeclareMathOperator{\dalambert}{\Box}
\DeclareMathOperator{\laplace}{\bigtriangleup}
\newcommand{\curl}{\vnabla \times}
\newcommand{\divergence}[1]{\inner{\vnabla}{#1}}
\newcommand{\vnabla}{\vec \nabla}

\newcommand{\half}{\frac 12}

% Unit vector (German „Einheitsvektor“).
\newcommand{\ev}{\hat{\vec e}}

% Scientific notation for large numbers.
\newcommand{\e}[1]{\cdot 10^{#1}}

% Mathematician's notation for the inner (scalar, dot) product.
\newcommand{\bracket}[1]{\left\langle #1 \right\rangle}
\newcommand{\inner}[2]{\bracket{#1, #2}}

% Placeholders.
\newcommand{\emesswert}{\del{\messwert \pm \messwert}}
\newcommand{\fehlt}{\textcolor{darkred}{Hier fehlen noch Inhalte.}}
\newcommand{\messwert}{\textcolor{blue}{\square}}
\newcommand{\punkte}{\phantom{xxxxx}}
\newcommand{\punktevon}[1]{\begin{flushright}/ #1\end{flushright}}

% Separator for equations on a single line.
\newcommand{\eqnsep}{,\quad}

% Quantum Mechanics
\usepackage{braket}

%%%%%%%%%%%%%%%%%%%%%%%%%%%%%%%%%%%%%%%%%%%%%%%%%%%%%%%%%%%%%%%%%%%%%%%%%%%%%%%
%                                  Headings                                   %
%%%%%%%%%%%%%%%%%%%%%%%%%%%%%%%%%%%%%%%%%%%%%%%%%%%%%%%%%%%%%%%%%%%%%%%%%%%%%%%

% This will set fancy headings to the top of the page. The page number will be
% accompanied by the total number of pages. That way, you will know if any page
% is missing.
%
% If you do not want this for your document, you can just use
% ``\pagestyle{plain}``.

\usepackage{scrpage2}

\pagestyle{scrheadings}
\automark{section}
\cfoot{\footnotesize{Seite \thepage\ / \pageref{LastPage}}}
\chead{}
\ihead{}
\ohead{\rightmark}
\setheadsepline{.4pt}

%%%%%%%%%%%%%%%%%%%%%%%%%%%%%%%%%%%%%%%%%%%%%%%%%%%%%%%%%%%%%%%%%%%%%%%%%%%%%%%
%                            Bibliography (BibTeX)                            %
%%%%%%%%%%%%%%%%%%%%%%%%%%%%%%%%%%%%%%%%%%%%%%%%%%%%%%%%%%%%%%%%%%%%%%%%%%%%%%%

\newcommand{\bibliographyfile}{../../central-bibtex/Central}
\bibliographystyle{apalike2}

%%%%%%%%%%%%%%%%%%%%%%%%%%%%%%%%%%%%%%%%%%%%%%%%%%%%%%%%%%%%%%%%%%%%%%%%%%%%%%%
%                                Abbreviations                                %
%%%%%%%%%%%%%%%%%%%%%%%%%%%%%%%%%%%%%%%%%%%%%%%%%%%%%%%%%%%%%%%%%%%%%%%%%%%%%%%

\newcommand{\dhabk}{\mbox{d.\,h.}}

%%%%%%%%%%%%%%%%%%%%%%%%%%%%%%%%%%%%%%%%%%%%%%%%%%%%%%%%%%%%%%%%%%%%%%%%%%%%%%%
%                                  Licences                                   %
%%%%%%%%%%%%%%%%%%%%%%%%%%%%%%%%%%%%%%%%%%%%%%%%%%%%%%%%%%%%%%%%%%%%%%%%%%%%%%%

\usepackage{ccicons}

\newcommand{\ccbysadetext}{%
	\begin{small}
		Dieses Werk bzw. Inhalt steht unter einer
		\href{http://creativecommons.org/licenses/by-sa/3.0/deed.de}{%
			Creative Commons Namensnennung - Weitergabe unter gleichen
		Bedingungen 3.0 Unported Lizenz}.
	\end{small}
}

\newcommand{\ccbysadetitle}{%
	Lizenz: \href{http://creativecommons.org/licenses/by-sa/3.0/deed.de}
	{CC-BY-SA 3.0 \ccbysa}
}


\usepackage{csquotes}
\usepackage[section]{placeins}
\usepackage{booktabs}
\usepackage{pdflscape}

\newcommand\versuchsnummer{402}
\DeclareMathOperator\std{std}

\newcommand\erklaerungFehlerNotation{%
    In dieser Notation bedeutet \num{1.234 +- 0.005}, dass der Wert
    $\num{1.234} \pm \num{0.005}$ ist. Die Ziffern in Klammern sind die
    Fehlerangabe. Um den Fehler zu erhalten, wird diese von rechts über die
    Zahl gelegt, alle anderen Stellen werden auf 0 gesetzt.
}

\ihead{physik412 – Versuch \versuchsnummer}
\ifoot{M. Ueding, L. Lemmer}

\subject{Praktikumsprotokoll}
\title{Elektronische Übergänge in Atomen}
\subtitle{physik412 – Versuch \versuchsnummer}
\author{
Martin Ueding \\
\small{\href{mailto:mu@martin-ueding.de}{mu@martin-ueding.de}}
\and
Lino Lemmer \\
\small{\href{mailto:lino.lemmer@uni-bonn.de}{s6lilemm@uni-bonn.de}}
}
\publishers{Tutor: Jens Peter}

\setcounter{secnumdepth}{4}
\setcounter{tocdepth}{3}

\begin{document}

\maketitle

\begin{abstract}
    Im ersten Teil dieses Versuchs betrachten wir mit einem Fabry-Pérot-Etalon
    die, durch den Zeeman-Effekt entstehende, Aufspaltung einer Spektrallinie
    von Cadmium. Aus dieser bestimmen wir das Bohrsche Magneton.

    Im zweiten Teil führen wir den Frank-Hertz-Versuch durch und ermitteln aus
    diesem die Energie des ${}^1\text{D}_2\to{}^1\text{P}_1$-Übergangs von
    Quecksilber.
\end{abstract}

\tableofcontents

\chapter{Theorie}

\section{Energieniveaus von Elektronen im Atom}

\subsection{Ohne äußeres magnetisches Feld}

\subsection{Mit äußerem magnetischem Feld}

\subsection{Niveau-Übergänge}

\section{Natürliche Linienbreite und Linienverbreiterung}

\section{Fabry-Pérot-Etalon}

\section{Polarisationsfilter und $\frac{\lambda}4$-Plättchen}

\section{Hall-Sonde}

\section{Frank-Hertz-Versuch}

\subsection{Aufbau und Funktionsweise}

\subsection{Termschema von Quecksilber}

\chapter{Aufbau}

\chapter{Zeeman-Effekt}

\section{Durchführung}

Der Strom, bei der in transversaler Konfiguration die Linien zu unterscheiden
sind, ist \SI{1.9}{\ampere}. In longitudinaler Konfiguration ist der Strom
\SI{1.0}{\ampere}. Die longitudinalen Linien sind deutlich dunkler.

\section{Auswertung}

\begin{figure}[htbp]
    \centering
    \includegraphics[width=\linewidth]{Alles.pdf}
    \caption{%
    }
    \label{fig:}
\end{figure}

\begin{figure}[htbp]
    \centering
    \includegraphics[width=\linewidth]{Zoom.pdf}
    \caption{%
    }
    \label{fig:}
\end{figure}

\section{Diskussion}

\chapter{Frank-Hertz-Versuch}

\section{Durchführung}

\section{Auswertung}

\section{Diskussion}

\printbibliography

\end{document}

% vim: et spell spelllang=de tw=79
